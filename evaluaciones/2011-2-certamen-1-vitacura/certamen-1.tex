\documentclass[11pt,spanish]{article}
\usepackage[utf8]{inputenc}
\usepackage{babel}
\usepackage{fullpage}
\usepackage{listings}
\usepackage{mathpazo}
\usepackage{enumitem}
\usepackage{courier}
\usepackage{xcolor}
\usepackage{textcomp}
\usepackage{amsmath}
\usepackage{amssymb}
\usepackage{tikz}
\usepackage{fancyhdr}
\usepackage{graphics}
\usepackage{array}

\newcommand{\titulo}{Certamen 1, sábado 8 de octubre de 2011}
\newcommand{\cc}[1]{\hfil\texttt{#1}\hfil}
\newcommand{\pond}[1]{[{\small\textbf{#1\%}}]}

\pagestyle{fancy}
\lhead{%
  {\Large\bfseries Programación---\titulo} \\
  Nombre: \nombre\hfill
  Rol:    \rol
  \vspace{2ex}
}
\chead{}\rhead{}\lfoot{}\cfoot{}\rfoot{}
\renewcommand{\headrulewidth}{0pt}
\addtolength{\headheight}{7ex}
\headsep=4ex


\newcommand{\onelinerule}{\rule[2.3ex]{0pt}{0pt}}
\newcommand{\twolinerule}{\rule[6.2ex]{0pt}{0pt}}
\newcommand{\respuesta}{\framebox[\textwidth]{\twolinerule}}
\newcommand{\nombre}{%
  \begin{tikzpicture}[xscale=.4,yscale=.7]
    \draw (0, 0) rectangle (22, 1);
  \end{tikzpicture}%
}
%\newcommand{\rol}   {\framebox[0.3\textwidth]{\onelinerule}}
\newcommand{\rol}{%
  \begin{tikzpicture}[xscale=.4,yscale=.7]
    \draw[gray!40] ( 0, 0) grid      ( 9, 1);
    \draw          ( 0, 0) rectangle ( 9, 1);
    \draw          (10, 0) rectangle (11, 1);
    \draw (9 + .2, .5) -- (10 - .2, .5);
  \end{tikzpicture}%
}
\newcommand{\li}{\lstinline}
\newcommand{\pond}[1]{[{\small\textbf{#1\%}}]}

\lstdefinelanguage{py}{%
  classoffset=0,%
    morekeywords={%
      False,class,finally,is,return,None,continue,for,lambda,try,%
      True,def,from,nonlocal,while,and,del,global,not,with,print,%
      as,elif,if,or,yield,assert,else,import,pass,break,except,in,raise},%
    keywordstyle=\color{black!80}\bfseries,%
  classoffset=1,
    morekeywords={int,float,str,abs,len,raw_input,exit,range,min,max,%
      set,dict,tuple,list,bool,complex,round,sum,all,any,zip,map,filter,%
      sorted,reversed,dir,file,frozenset,open,%
      array,zeros,ones,arange,linspace,eye,diag,dot},
    keywordstyle=\color{black!50}\bfseries,%
  classoffset=0,%
  sensitive=true,%
  morecomment=[l]\#,%
  morestring=[b]',%
  morestring=[b]",%
  stringstyle=\em,%
}

\lstdefinelanguage{testcase}{%
  moredelim=[is][\bfseries]{`}{`},%
  backgroundcolor=\color{gray!20},%
}

\lstdefinelanguage{file}{%
  frame=single,%
  backgroundcolor=\color{white},%
}

\lstset{language=py}
\lstset{basicstyle=\ttfamily}
\lstset{columns=fixed}
\lstset{upquote=true}
\lstset{showstringspaces=false}
\lstset{rangeprefix=\#\ }
\lstset{includerangemarker=false}

\newlist{certamen}{enumerate}{1}
\setlist[certamen]{%
  label=\arabic*.,
  font=\LARGE\bfseries,%
  labelindent=-.5in,%
  leftmargin=0pt,%
  labelsep=1em%
}



\begin{document}

  \begin{enumerate}[font=\Large\bfseries]

    % Entender programas
    \item%[1a.]
      \pond{25}
      Indique qué es lo que imprimen los siguientes programas.

      \foreach \x in {1,2} {
        \noindent
        \begin{minipage}[b]{.5\textwidth}
          \lstinputlisting{p\x.py}
          \framebox[.8\textwidth]{\rule[10ex]{0pt}{0pt}}
          \vspace{0.4em}
        \end{minipage}
      }

    %\item[1b.]
      Rutee el siguiente programa
      e indique qué es lo que imprime.

      Cada vez que el valor de una variable cambie,
      ponga su valor en una nueva fila de la tabla.
      La tabla tiene filas de sobra.

      \begin{minipage}[T]{.5\textwidth}
        \lstinputlisting{ruteo.py}
        \framebox[.8\textwidth]{\rule[10ex]{0pt}{0pt}}
      \end{minipage}
      \begin{minipage}[t]{.4\textwidth}\centering
        \begin{tabular}{|*{5}{p{2.6em}|}}\hline
            \cc{a} & \cc{c} & \cc{x} & \cc{y} & \cc{z} \\ \hline\hline
            &&&& \\\hline &&&& \\\hline &&&& \\\hline &&&& \\\hline &&&& \\\hline
            &&&& \\\hline &&&& \\\hline &&&& \\\hline &&&& \\\hline &&&& \\\hline
            &&&& \\\hline &&&& \\\hline &&&& \\\hline &&&& \\\hline &&&& \\\hline
            &&&& \\\hline &&&& \\\hline &&&& \\\hline &&&& \\\hline &&&& \\\hline
            &&&& \\\hline &&&& \\\hline &&&& \\\hline &&&& \\\hline &&&& \\\hline
         \end{tabular}
      \end{minipage}

    \newpage
    \item
      \pond{25}
      La Universidad Tropical Filomena Santa Marta
      ha instaurado un nuevo reglamento de evaluaciones.
      Todas las asignaturas deben tener tres certámenes y un examen.
      Las notas van entre 0 y 10, con un decimal.

      Después de los tres certámenes,
      los alumnos con promedio menor que 3 reprueban y
      los con promedio mayor o igual que 7 aprueban.
      El resto va al examen,
      en el que deben sacarse por lo menos un 5 para aprobar.

      Además,
      para reducir el trabajo de los profesores,
      se decidió que los alumnos que se sacan menos de un 2
      en los dos primeros certámenes
      están automáticamente reprobados.
      A su vez,
      los que obtienen más de un 9
      en los dos primeros certámenes
      están automáticamente aprobados.
      En ambos casos,
      no deben rendir el tercer certamen.

      Escriba un programa que pregunte a un estudiante
      las notas de las evaluaciones que rindió,
      y le diga si está aprobado o reprobado.

      \begin{minipage}[t]{.2\textwidth}
        \lstinputlisting[language=testcase,frame=single,linerange=1-4]{caso-utfsm.txt}
      \end{minipage}
      \hfil
      \begin{minipage}[t]{.2\textwidth}
        \lstinputlisting[language=testcase,frame=single,linerange=5-8]{caso-utfsm.txt}
      \end{minipage}
      \hfil
      \begin{minipage}[t]{.2\textwidth}
        \lstinputlisting[language=testcase,frame=single,linerange=10-14]{caso-utfsm.txt}
      \end{minipage}
      %\hfil
      %\begin{minipage}[t]{.2\textwidth}
      %  \lstinputlisting[language=testcase,frame=single,linerange=16-20]{caso-utfsm.txt}
      %\end{minipage}
      %\hfil
      %\begin{minipage}[t]{.2\textwidth}
      %  \lstinputlisting[language=testcase,frame=single,linerange=22-25]{caso-utfsm.txt}
      %\end{minipage}
      \hfil
      \begin{minipage}[t]{.2\textwidth}
        \lstinputlisting[language=testcase,frame=single,linerange=27-29]{caso-utfsm.txt}
      \end{minipage}

    \newpage
    \item
      \pond{25}
      La CONFECH,
      en su afán de agilizar el proceso de recuento de las votaciones,
      le ha encargado el desarrollo de un programa de registro de votación por universidades.

      \begin{minipage}[t]{.5\textwidth}
        Primero, el programa debe solicitar al usuario ingresar
        la cantidad de universidades que participan en el proceso.
        \vspace{1ex}

        Luego, para cada una de las universidades,
        el usuario debe ingresar
        el nombre de la universidad
        y los votos de sus alumnos, que pueden ser:
        \emph{aceptar} (\texttt{A}),
        \emph{rechazar} (\texttt{R}),
        \emph{nulo} (\texttt{N}) o
        \emph{blanco} (\texttt{B}).
        El término de la votación se indica ingresando una \texttt{X},
        tras lo cual se debe mostrar los totales de votos de la universidad,
        con el formato que se muestra en el ejemplo.
        \vspace{1ex}

        Finalmente,
        el programa debe mostrar el resultado de la votación,
        indicando la cantidad de universidades que aceptan, que rechazan
        y en las que hubo empate entre estas dos opciones.

      \end{minipage}
      \hfill
      \begin{minipage}[t]{.4\textwidth}
        \lstinputlisting[language=testcase,frame=single,basicstyle=\tiny\ttfamily]{caso-confech.txt}
      \end{minipage}

    \newpage
    \item
      \pond{25}
      El supermercado Pitón Market ha lanzado una promoción
      para todos sus clientes que posean la tarjeta Raw Input.
      La promoción consiste en aplicar un descuento
      por cada \(n\) productos que pasan por caja.

      \begin{minipage}[t]{.55\textwidth}
        El primer descuento es de 20\%,
        y se aplica sobre los primeros \(n\) productos ingresados.
        Luego,
        cada descuento es la mitad del anterior,
        y es aplicado sobre los siguientes \(n\) productos.
        \vspace{1ex}

        Por ejemplo,
        si \(n = 3\) y la compra es de 11 productos,
        entonces los tres primeros tienen 20\% de descuento,
        los tres siguientes 10\%,
        los tres siguientes 5\%,
        y los dos últimos no tienen descuento.
        \vspace{1ex}

        Escriba un programa que pida al usuario ingresar \(n\)
        y la cantidad de productos,
        y luego los precios de cada producto.
        Al final,
        el programa debe entregar el precio total,
        el descuento total
        y el precio final después de aplicar el descuento.
        \vspace{1ex}

        Si al aplicar el descuento el precio queda con decimales,
        redondee el valor hacia abajo.
      \end{minipage}
      \hfill
      \begin{minipage}[t]{.35\textwidth}
        \lstinputlisting[language=testcase,frame=single]{caso-descuento.txt}
      \end{minipage}

  \end{enumerate}
\end{document}

