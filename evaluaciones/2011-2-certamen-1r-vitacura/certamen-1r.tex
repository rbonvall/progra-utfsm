\documentclass[11pt,spanish]{article}
\usepackage[utf8]{inputenc}
\usepackage{babel}
\usepackage{fullpage}
\usepackage{listings}
\usepackage{mathpazo}
\usepackage{enumitem}
\usepackage{courier}
\usepackage{xcolor}
\usepackage{textcomp}
\usepackage{amsmath}
\usepackage{amssymb}
\usepackage{tikz}
\usepackage{fancyhdr}
\usepackage{graphics}
\usepackage{array}
\usetikzlibrary{calc,patterns,arrows}

\newcommand{\titulo}{Recuperativo 1, xxxxx XX de noviembre de 2011}
\newcommand{\cc}[1]{\hfil\texttt{#1}\hfil}
\newcommand{\pond}[1]{[{\small\textbf{#1\%}}]}

\pagestyle{fancy}
\lhead{%
  {\Large\bfseries Programación---\titulo} \\
  Nombre: \nombre\hfill
  Rol:    \rol
  \vspace{2ex}
}
\chead{}\rhead{}\lfoot{}\cfoot{}\rfoot{}
\renewcommand{\headrulewidth}{0pt}
\addtolength{\headheight}{7ex}
\headsep=4ex


\newcommand{\onelinerule}{\rule[2.3ex]{0pt}{0pt}}
\newcommand{\twolinerule}{\rule[6.2ex]{0pt}{0pt}}
\newcommand{\respuesta}{\framebox[\textwidth]{\twolinerule}}
\newcommand{\nombre}{%
  \begin{tikzpicture}[xscale=.4,yscale=.7]
    \draw (0, 0) rectangle (22, 1);
  \end{tikzpicture}%
}
%\newcommand{\rol}   {\framebox[0.3\textwidth]{\onelinerule}}
\newcommand{\rol}{%
  \begin{tikzpicture}[xscale=.4,yscale=.7]
    \draw[gray!40] ( 0, 0) grid      ( 9, 1);
    \draw          ( 0, 0) rectangle ( 9, 1);
    \draw          (10, 0) rectangle (11, 1);
    \draw (9 + .2, .5) -- (10 - .2, .5);
  \end{tikzpicture}%
}
\newcommand{\li}{\lstinline}
\newcommand{\pond}[1]{[{\small\textbf{#1\%}}]}

\lstdefinelanguage{py}{%
  classoffset=0,%
    morekeywords={%
      False,class,finally,is,return,None,continue,for,lambda,try,%
      True,def,from,nonlocal,while,and,del,global,not,with,print,%
      as,elif,if,or,yield,assert,else,import,pass,break,except,in,raise},%
    keywordstyle=\color{black!80}\bfseries,%
  classoffset=1,
    morekeywords={int,float,str,abs,len,raw_input,exit,range,min,max,%
      set,dict,tuple,list,bool,complex,round,sum,all,any,zip,map,filter,%
      sorted,reversed,dir,file,frozenset,open,%
      array,zeros,ones,arange,linspace,eye,diag,dot},
    keywordstyle=\color{black!50}\bfseries,%
  classoffset=0,%
  sensitive=true,%
  morecomment=[l]\#,%
  morestring=[b]',%
  morestring=[b]",%
  stringstyle=\em,%
}

\lstdefinelanguage{testcase}{%
  moredelim=[is][\bfseries]{`}{`},%
  backgroundcolor=\color{gray!20},%
}

\lstdefinelanguage{file}{%
  frame=single,%
  backgroundcolor=\color{white},%
}

\lstset{language=py}
\lstset{basicstyle=\ttfamily}
\lstset{columns=fixed}
\lstset{upquote=true}
\lstset{showstringspaces=false}
\lstset{rangeprefix=\#\ }
\lstset{includerangemarker=false}

\newlist{certamen}{enumerate}{1}
\setlist[certamen]{%
  label=\arabic*.,
  font=\LARGE\bfseries,%
  labelindent=-.5in,%
  leftmargin=0pt,%
  labelsep=1em%
}



\begin{document}

  \begin{enumerate}[font=\Large\bfseries]

    % Entender programas
    \item%[1a.]
      \pond{25}
      Indique qué es lo que imprimen los siguientes programas.

      \foreach \x in {1,2} {
        \noindent
        \begin{minipage}[b]{.5\textwidth}
          \lstinputlisting{p\x.py}
          \framebox[.8\textwidth]{\rule[10ex]{0pt}{0pt}}
          \vspace{0.4em}
        \end{minipage}
      }

    %\item[1b.]
      Rutee el siguiente programa
      e indique qué es lo que imprime.

      Cada vez que el valor de una variable cambie,
      ponga su valor en una nueva fila de la tabla.
      La tabla tiene filas de sobra.

      \begin{minipage}[T]{.5\textwidth}
        \lstinputlisting{ruteo.py}
        \framebox[.8\textwidth]{\rule[10ex]{0pt}{0pt}}
      \end{minipage}
      \begin{minipage}[t]{.4\textwidth}\centering
        \begin{tabular}{|*{5}{p{2.6em}|}}\hline
            \cc{a} & \cc{c} & \cc{x} & \cc{y} & \cc{z} \\ \hline\hline
            &&&& \\\hline &&&& \\\hline &&&& \\\hline &&&& \\\hline &&&& \\\hline
            &&&& \\\hline &&&& \\\hline &&&& \\\hline &&&& \\\hline &&&& \\\hline
            &&&& \\\hline &&&& \\\hline &&&& \\\hline &&&& \\\hline &&&& \\\hline
            &&&& \\\hline &&&& \\\hline &&&& \\\hline &&&& \\\hline &&&& \\\hline
            &&&& \\\hline &&&& \\\hline &&&& \\\hline &&&& \\\hline &&&& \\\hline
         \end{tabular}
      \end{minipage}

    \newpage
    \item
      \pond{25}

    \newpage
    \item
      \pond{25}

    \newpage
    \item
      \pond{25}
      Una circunferencia en el plano está definida por tres valores:
      las coordenadas \((x, y)\) de su centro, y su radio \(r\).

      Escriba un programa que determine si dos circunferencias
      se intersectan o no.

      \def\grilla{%
        \draw[gray] (0, 0) grid (14, 10);
        \draw[-latex'] (0, 0) to (15, 0);
        \draw[-latex'] (0, 0) to (0, 11);
      }
      \newcommand\circulo[3]{%
        \node[inner sep=1pt, fill, circle] (p1) at (#1, #2) {};
        \draw[very thick] (p1) circle (#3 cm);
      }

      \begin{tikzpicture}[scale=.3]
        \grilla
        \circulo{ 5}{6}{3.5}
        \circulo{10}{5}{3}
      \end{tikzpicture}
      \hfill
      \begin{tikzpicture}[scale=.3]
        \grilla
        \circulo{3.5}{5}{2}
        \circulo{10}{4}{3}
      \end{tikzpicture}
      \hfill
      \begin{tikzpicture}[scale=.3]
        \grilla
        \circulo{5}{4.5}{3}
        \circulo{6}{5}{4.5}
      \end{tikzpicture}

      \begin{minipage}[t]{.28\textwidth}
        \lstinputlisting[language=testcase,frame=single,linerange=CASO\ 1-FIN\ CASO\ 1]{caso-circ.txt}
      \end{minipage}
      \hfill
      \begin{minipage}[t]{.28\textwidth}
        \lstinputlisting[language=testcase,frame=single,linerange=CASO\ 2-FIN\ CASO\ 2]{caso-circ.txt}
      \end{minipage}
      \hfill
      \begin{minipage}[t]{.28\textwidth}
        \lstinputlisting[language=testcase,frame=single,linerange=CASO\ 3-FIN\ CASO\ 3]{caso-circ.txt}
      \end{minipage}

  \end{enumerate}
\end{document}

