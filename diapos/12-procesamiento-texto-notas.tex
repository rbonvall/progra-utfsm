\documentclass[10pt]{article}
\usepackage{beamerarticle}
\usepackage[spanish]{babel}
\usepackage[utf8]{inputenc}
\usepackage{fullpage}
\usepackage{xcolor}
\usepackage{listings}
\usepackage{textcomp}
\usepackage{mathpazo}
\usepackage{courier}
\usepackage{fancyvrb}
\usepackage{amsmath}
\usepackage{url}
\usepackage{hyperref}
\usepackage{pgfpages}
\usepackage{wrapfig}
\usepackage{enumitem}

\setjobnamebeamerversion{12-procesamiento-texto-diapos}

\newcommand{\onelinerule}{\rule[2.3ex]{0pt}{0pt}}
\newcommand{\twolinerule}{\rule[6.2ex]{0pt}{0pt}}
\newcommand{\respuesta}{\framebox[\textwidth]{\twolinerule}}
\newcommand{\nombre}{%
  \begin{tikzpicture}[xscale=.4,yscale=.7]
    \draw (0, 0) rectangle (22, 1);
  \end{tikzpicture}%
}
%\newcommand{\rol}   {\framebox[0.3\textwidth]{\onelinerule}}
\newcommand{\rol}{%
  \begin{tikzpicture}[xscale=.4,yscale=.7]
    \draw[gray!40] ( 0, 0) grid      ( 9, 1);
    \draw          ( 0, 0) rectangle ( 9, 1);
    \draw          (10, 0) rectangle (11, 1);
    \draw (9 + .2, .5) -- (10 - .2, .5);
  \end{tikzpicture}%
}
\newcommand{\li}{\lstinline}
\newcommand{\pond}[1]{[{\small\textbf{#1\%}}]}

\lstdefinelanguage{py}{%
  classoffset=0,%
    morekeywords={%
      False,class,finally,is,return,None,continue,for,lambda,try,%
      True,def,from,nonlocal,while,and,del,global,not,with,print,%
      as,elif,if,or,yield,assert,else,import,pass,break,except,in,raise},%
    keywordstyle=\color{black!80}\bfseries,%
  classoffset=1,
    morekeywords={int,float,str,abs,len,raw_input,exit,range,min,max,%
      set,dict,tuple,list,bool,complex,round,sum,all,any,zip,map,filter,%
      sorted,reversed,dir,file,frozenset,open,%
      array,zeros,ones,arange,linspace,eye,diag,dot},
    keywordstyle=\color{black!50}\bfseries,%
  classoffset=0,%
  sensitive=true,%
  morecomment=[l]\#,%
  morestring=[b]',%
  morestring=[b]",%
  stringstyle=\em,%
}

\lstdefinelanguage{testcase}{%
  moredelim=[is][\bfseries]{`}{`},%
  backgroundcolor=\color{gray!20},%
}

\lstdefinelanguage{file}{%
  frame=single,%
  backgroundcolor=\color{white},%
}

\lstset{language=py}
\lstset{basicstyle=\ttfamily}
\lstset{columns=fixed}
\lstset{upquote=true}
\lstset{showstringspaces=false}
\lstset{rangeprefix=\#\ }
\lstset{includerangemarker=false}

\newlist{certamen}{enumerate}{1}
\setlist[certamen]{%
  label=\arabic*.,
  font=\LARGE\bfseries,%
  labelindent=-.5in,%
  leftmargin=0pt,%
  labelsep=1em%
}


\newcommand{\diapo}[1]{
  %\begin{wrapfigure}[11]{l}{.35\textwidth}
    \vspace{2ex}
    \fbox{\includeslide[width=.3\textwidth]{#1}}
  %\end{wrapfigure}
}


\title{Procesamiento de texto}
\author{Programación \\ \url{http://progra.usm.cl}}
\date{}

\begin{document}
  \maketitle

  \section*{Objetivos de la clase}
  \begin{itemize}
    \item Enseñar algunas operaciones típicas
      sobre strings.
    \item Mostrar aplicaciones reales de estas operaciones.
  \end{itemize}

  \section*{Diapositivas}

  No todas las operaciones sobre texto
  serán vistas en la clase,
  ya que son muchas y el tiempo es poco.
  En las diapositivas sólo aparecen
  las que serán de utilidad en materias futuras
  (especialmente para archivos de texto).

  Algunas operaciones más
  están cubiertas en \href{http://progra.usm.cl/materia/texto.html}{la página de la materia}.

  Para ver todos los métodos que tiene el tipo string,
  puede ingresar \li!dir(str)! en la consola.

  \diapo{saltos-linea}

  Los saltos de línea deben ser tenidos en cuenta
  al leer y escribir en archivos,
  por lo que es necesario mencionarlos en esta clase.

  Un \emph{salto de línea}
  es un caracter que indica que el texto inmediatamente a continuación
  debe ser impreso en una nueva línea.
  Para que esto sea más fácil de entender para los estudiantes,
  puede decir que es el equivalente a presionar la tecla Enter.

  En Python, dentro de un string, el salto de línea se representa
  como \li!\n!, al igual que en C y Java.

  A pesar de que su representación en el string usa dos símbolos,
  el salto de línea es un único caracter.
  Enfatice esto,
  usando el ejemplo de abajo en la diapositiva como evidencia.

  Los saltos de línea sólo aparecen como tales
  al imprimir el string usando la sentencia \li!print!.
  Al mostrar el string en la consola,
  aparecerá representado como \li!\n!.

  Además del salto de línea,
  dentro de un string pueden ir otros caracteres especiales,
  como la tabulación (\li!\t!), el retroceso (\li!\b!)
  y las comillas escapadas (\li!\'!, \li!\"!), entre otros.
  Éstos no serán considerados parte de la materia del ramo.

  \diapo{reemplazar}

  El método \li!replace!
  reemplaza las ocurrencias de un substring por otro.
  Explique los ejemplos uno por uno.

  Si bien la operación es fácil de entender,
  hay que tener claro que no modifica el string original,
  sino que retorna uno nuevo.

  \diapo{partir-string-palabras}

  El método \li!split! llamado sin argumentos
  separa el string en una lista de strings,
  partiéndolo en los espacios en blanco.

  Un caso de uso típico
  es obtener las palabras de una oracion: \li!palabras = oracion.split()!.

  Otro caso de uso típico
  es parsear varios datos ingresados como entrada en una sola línea
  mediante una única llamada a \li!raw_input!:
  \li!datos = entrada.split()!.
  En este caso, suele ser necesario convertir los datos al tipo apropiado.

  En la diapositiva,
  además se muestra como separar una oración por letras en vez de palabras.
  Para esto, basta con usar la función \li!list!.

  El método \li!split! usa regiones de espacios para partir la oración,
  no espacios individuales.
  Esto significa que \li!'a    b       c'.split()!
  es lo mismo que \li!'a b c'.split()!.

  Mencione que el valor de retorno del método \li!split!
  siempre es de tipo lista (\li!list!).

  \diapo{partir-string-separador}

  La otra manera de usar el método \li!split!
  es pasar explícitamente cuál es la cadena usada como separador.

  En el ejemplo de la diapositiva, a modo de comparación,
  se muestra el mismo string separado por espacios en blanco,
  por el caracter \li!'a'! y por el string \li!'ac'!.
  Explique en detalle de dónde vienen los resultados de cada uno
  de los ejemplos.

  A diferencia del \li!split! sin argumento,
  el string es partido en las apariciones individuales del separador,
  no por regiones.
  Esto significa que \li+'a::b:c'.split(':')+
  \textbf{no} es lo mismo que \li+'a:b::::c'.split(':')+.

  Note que si el separador aparece al principio o al final del string,
  o si aparece dos veces sin nada entre medio,
  aparecerá un string vacío en la lista de ítemes.
  Esto se puede ver en el segundo ejemplo.

  El caso de uso típico de esta forma del método \li!split!
  es parsear datos de un archivo CSV (materia de la próxima clase).

  \diapo{unir-strings}

  La operación inversa del \li!split! es el \li!join!,
  que crea un único string a partir de una lista de strings,
  pegándolos con un separador dado.

  Hay que recordar siempre
  que el método \li!join! debe ser invocado sobre el string separador,
  y no sobre los valores a pegar.
  Esto se debe a que es un método de la clase \li!str!, no de \li!list!.

  En los ejemplos de la diapositiva,
  las palabras de la lista son pegadas
  usando un espacio, usando el string vacío (útil para concatenar strings)
  y un string con más de un caracter.

  \diapo{unir-valores}

  El método \li!join! recibe como parámetro una lista de strings.
  Si es llamado con una lista de valores de otro tipo,
  ocurrirá un \li!TypeError!.

  Pegar varios valores en un string es un caso de uso frecuente,
  por lo que hay que tener cuidado de no cometer este error.
  La manera más simple de evitarlo
  es convertir a string todos los elementos de la lista
  usando la función \li!map!.

  La función \li!map(f, l)!
  aplica la función \li!f! a todos los elementos de la lista \li!l!.
  No modifica la lista, sino que retorna una nueva:
\begin{lstlisting}
>>> map(abs, [-3, 2, -1, 0])
[3, 2, 1, 0]
>>> map(str, range(3))
['0', '1', '2']
\end{lstlisting}
  Explique la función \li!map! mostrando un par de ejemplos en la consola.
  Es realmente conveniente,
  ya que evita la necesidad de crear un ciclo sólo para convertir los valores.

  \diapo{interpolacion-posicion}

  La interpolación consiste en usar un string como una plantilla,
  en la que se rellenan los valores usando el método \li!format!.
  Es equivalente en funcionalidad a la familia de funciones \li!printf! de C.

  (El nombre «interpolación» no es muy usado en Python,
  pero sí es común en otros lenguajes como Perl y PHP).

  Los campos a rellenar se representan usando números entre llaves.
  Cada número se refiere a uno de los parámetros que se debe pasar
  al método \li!format!. Los campos pueden aparecer más de una vez,
  y pueden estar en cualquier orden dentro del string:
\begin{lstlisting}
>>> '{2}{0}{1}{0}'.format(3, 'ch', 'l')
'l3ch3'
\end{lstlisting}

  \diapo{interpolacion-nombre}

  Además de indicar los campos por posición,
  también es posible hacerlo por nombre.
  En este caso,
  los parámetros del método \li!format!
  deben ser pasados explícitamente por nombre.

  Como información al margen que no es necesario mencionar en clases:
  dentro de las llaves de los campos
  es posible especificar cómo debe ser mostrado el valor
  usando algunos modificadores.
  Por ejemplo, para mostrar el valor de \(\pi\)
  alineado a la sexta columna a la derecha
  y mostrando sólo dos decimales,
  puede hacerse así:
\begin{lstlisting}
>>> '{0:>6.2f}'.format(3.14159)
'  3.14'
\end{lstlisting}

  \diapo{mayusculas-minusculas}

  Los métodos \li!upper!, \li!lower! y \li!swapcase!
  sirven para convertir el string a mayúsculas, a minúsculas
  y para cambiar entre unas y otras, respectivamente.

  El caso de uso típico de estos métodos
  es normalizar strings para que todas las palabras repetidas
  aparezcan de la misma forma en un texto.

  \diapo{problema-adn-enunciado}
  \diapo{problema-adn-aleatorio}

  (Resuelva los ejercicios al ritmo de sus alumnos.
  No es grave que no alcance a resolverlos todos).

  (Los ejercicios no aparecen resueltos en las diapositivas,
  sino en los archivos adjuntos,
  para que sea más fácil probarlos y mostrarlos en funcionamiento).

  Lea el enunciado del ejercicio en voz alta.

  El primer ejercicio es crear una función
  que retorne una cadena aleatoria.
  Como pista, se muestra cómo obtener una única base
  (\li!'a'!, \li!'t'!, \li!'c'! o \li!'g'!)
  al azar, usando la función \li!choice! del módulo \li!random!.

  Procure que no cause confusión el hecho que la misma función
  retorna resultados distintos al llamarla varias veces con el mismo parámetro.

  La solución consiste en ir generando bases individuales,
  e ir metiéndolas a una lista.
  Como el resultado debe ser una cadena,
  al final hay que pegarlas con el método \li!join!.

  \diapo{problema-adn-complementaria}

  Lea el enunciado en voz alta, y explique lo que hay que hacer.
  Hay (por lo menos) dos estrategias para resolver este problema:
  \begin{enumerate}
    \item hacer el reemplazo usando el método \li!replace!, y
    \item crear una lista con las bases de la cadena complementaria,
      y al final pegarla con el método \li!join!.
  \end{enumerate}

  Espere que a los alumnos se les ocurra alguna de estas estrategias
  e impleméntela junto con ellos.

  La primera estrategia presenta una dificultad:
  no basta con reemplazar \li!'a'! por \li!'t'! y \li!'t'! por \li!'a'!,
  ya que si se hace así al final quedará sólo una de las letras.
  Lo que hay que hacer es primero reemplazar una de las letras por un caracter neutral.
  Intente que los alumnos descubran este problema por sí solos.

  El archivo adjunto presenta varias implementaciones para la solución de este ejercicio.

  \diapo{problema-reporte-promedios}

  Explique el ejercicio usando el ejemplo como referencia.
  El algoritmo a usar no tiene nada de nuevo,
  pero hay que preocuparse de los detalles:
  \begin{itemize}
    \item las notas se ingresan en una única línea,
      por lo que hay que parsear la entrada
      usando \li!split! y luego convertir a un tipo numérico;
    \item el programa pregunta las notas
      usando el nombre (sin apellido) del alumno,
      por lo que hay que partir el nombre completo con \li!split!
      y quedarse con el primero de los valores;
    \item los reportes se muestran al final,
      por lo que hay que guardar los datos usando listas.
  \end{itemize}

  Los formatos de la entrada y la salida
  son una excelente oportunidad para usar el método \li!format!.

  En la solución adjunta,
  se usa los modificadores de los campos
  para mostrar sólo dos decimales del promedio final.

  \diapo{problema-cuenta-palabras}

  Lea el enunciado en voz alta.

  El problema principal radica en deshacerse de los signos de puntuación
  y en normalizar el string para que todas las palabras iguales
  aparezcan de la misma forma, todo esto antes de partir la oración en palabras.

  De ahí en adelante,
  el problema se reduce al patrón de crear un diccionario de contadores,
  que ya hemos presentado en clases anteriores.

  Si le sobra tiempo,
  puede preguntar a los alumnos cómo harían para crear un diccionario
  que cuente las letras en lugar de las palabras.
  (Hay que normalizar a minúsculas, deshacerse de los espacios en blanco
  además de la puntuación, y no usar \li!split!).

\end{document}

