\documentclass[10pt]{article}
\usepackage{beamerarticle}
\usepackage[spanish]{babel}
\usepackage[utf8]{inputenc}
\usepackage{fullpage}
\usepackage{xcolor}
\usepackage{listings}
\usepackage{textcomp}
\usepackage{mathpazo}
\usepackage{courier}
\usepackage{fancyvrb}
\usepackage{amsmath}
\usepackage{url}
\usepackage{hyperref}
\usepackage{pgfpages}

\setjobnamebeamerversion{02-diapos}

\newcommand{\onelinerule}{\rule[2.3ex]{0pt}{0pt}}
\newcommand{\twolinerule}{\rule[6.2ex]{0pt}{0pt}}
\newcommand{\respuesta}{\framebox[\textwidth]{\twolinerule}}
\newcommand{\nombre}{%
  \begin{tikzpicture}[xscale=.4,yscale=.7]
    \draw (0, 0) rectangle (22, 1);
  \end{tikzpicture}%
}
%\newcommand{\rol}   {\framebox[0.3\textwidth]{\onelinerule}}
\newcommand{\rol}{%
  \begin{tikzpicture}[xscale=.4,yscale=.7]
    \draw[gray!40] ( 0, 0) grid      ( 9, 1);
    \draw          ( 0, 0) rectangle ( 9, 1);
    \draw          (10, 0) rectangle (11, 1);
    \draw (9 + .2, .5) -- (10 - .2, .5);
  \end{tikzpicture}%
}
\newcommand{\li}{\lstinline}
\newcommand{\pond}[1]{[{\small\textbf{#1\%}}]}

\lstdefinelanguage{py}{%
  classoffset=0,%
    morekeywords={%
      False,class,finally,is,return,None,continue,for,lambda,try,%
      True,def,from,nonlocal,while,and,del,global,not,with,print,%
      as,elif,if,or,yield,assert,else,import,pass,break,except,in,raise},%
    keywordstyle=\color{black!80}\bfseries,%
  classoffset=1,
    morekeywords={int,float,str,abs,len,raw_input,exit,range,min,max,%
      set,dict,tuple,list,bool,complex,round,sum,all,any,zip,map,filter,%
      sorted,reversed,dir,file,frozenset,open,%
      array,zeros,ones,arange,linspace,eye,diag,dot},
    keywordstyle=\color{black!50}\bfseries,%
  classoffset=0,%
  sensitive=true,%
  morecomment=[l]\#,%
  morestring=[b]',%
  morestring=[b]",%
  stringstyle=\em,%
}

\lstdefinelanguage{testcase}{%
  moredelim=[is][\bfseries]{`}{`},%
  backgroundcolor=\color{gray!20},%
}

\lstdefinelanguage{file}{%
  frame=single,%
  backgroundcolor=\color{white},%
}

\lstset{language=py}
\lstset{basicstyle=\ttfamily}
\lstset{columns=fixed}
\lstset{upquote=true}
\lstset{showstringspaces=false}
\lstset{rangeprefix=\#\ }
\lstset{includerangemarker=false}

\newlist{certamen}{enumerate}{1}
\setlist[certamen]{%
  label=\arabic*.,
  font=\LARGE\bfseries,%
  labelindent=-.5in,%
  leftmargin=0pt,%
  labelsep=1em%
}



\title{Uso de intérprete, expresiones y tipos}
\author{Programación \\ \url{http://progra.usm.cl}}
\date{9 y 10 de marzo de 2011}

\begin{document}
  \maketitle

  \includeslide[width=.5\textwidth]{prog-ejemplo}

  Ésta diapositiva presenta el programa de ejemplo
  que será usado para mostrar cómo ejecutar un programa.
  Además, introduce algunas de las convenciones
  que serán seguidas durante todo el semestre
  en las diapositivas, los apuntes y las evaluaciones:
  \begin{itemize}
    \item los enunciados explican el problema en palabras;
    \item el caso de prueba muestra cómo se ve la ejecución de un programa,
      donde la entrada ingresada por el usuario aparece en negrita;
    \item el código es el programa que puede ser copiado y pegado
      en un editor de texto para ser guardado y ejecutado.
  \end{itemize}

  En este punto,
  explicar en detalle qué hace cada parte del programa,
  para que los estudiantes comiencen a familiarizarse con el código.
  En particular, explicar lo siguiente:
  \begin{itemize}
    \item la primera línea del programa es la entrada,
      la segunda línea es el proceso
      y la tercera línea es la salida;
    \item la función \li!raw_input! recibe como parámetro
      el texto que será mostrado en la pantalla,
      y entrega com resultado el texto ingresado por el usuario;
    \item la función \li!float! convierte el texto a un número real,
      lo que es necesario para poder hacer operaciones aritméticas
      con el valor ingresado;
    \item cómo es evaluada la expresión,
      y su resultado asignado a la variable \li!c!;
  \end{itemize}

\end{document}


