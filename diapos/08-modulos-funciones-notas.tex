\documentclass[10pt]{article}
\usepackage{beamerarticle}
\usepackage[spanish]{babel}
\usepackage[utf8]{inputenc}
\usepackage{fullpage}
\usepackage{xcolor}
\usepackage{listings}
\usepackage{textcomp}
\usepackage{mathpazo}
\usepackage{courier}
\usepackage{fancyvrb}
\usepackage{amsmath}
\usepackage{url}
\usepackage{hyperref}
\usepackage{pgfpages}
\usepackage{wrapfig}
\usepackage{enumitem}

\setjobnamebeamerversion{08-modulos-funciones-diapos}

\hyphenation{re-fe-ren-cia}

\newcommand{\onelinerule}{\rule[2.3ex]{0pt}{0pt}}
\newcommand{\twolinerule}{\rule[6.2ex]{0pt}{0pt}}
\newcommand{\respuesta}{\framebox[\textwidth]{\twolinerule}}
\newcommand{\nombre}{%
  \begin{tikzpicture}[xscale=.4,yscale=.7]
    \draw (0, 0) rectangle (22, 1);
  \end{tikzpicture}%
}
%\newcommand{\rol}   {\framebox[0.3\textwidth]{\onelinerule}}
\newcommand{\rol}{%
  \begin{tikzpicture}[xscale=.4,yscale=.7]
    \draw[gray!40] ( 0, 0) grid      ( 9, 1);
    \draw          ( 0, 0) rectangle ( 9, 1);
    \draw          (10, 0) rectangle (11, 1);
    \draw (9 + .2, .5) -- (10 - .2, .5);
  \end{tikzpicture}%
}
\newcommand{\li}{\lstinline}
\newcommand{\pond}[1]{[{\small\textbf{#1\%}}]}

\lstdefinelanguage{py}{%
  classoffset=0,%
    morekeywords={%
      False,class,finally,is,return,None,continue,for,lambda,try,%
      True,def,from,nonlocal,while,and,del,global,not,with,print,%
      as,elif,if,or,yield,assert,else,import,pass,break,except,in,raise},%
    keywordstyle=\color{black!80}\bfseries,%
  classoffset=1,
    morekeywords={int,float,str,abs,len,raw_input,exit,range,min,max,%
      set,dict,tuple,list,bool,complex,round,sum,all,any,zip,map,filter,%
      sorted,reversed,dir,file,frozenset,open,%
      array,zeros,ones,arange,linspace,eye,diag,dot},
    keywordstyle=\color{black!50}\bfseries,%
  classoffset=0,%
  sensitive=true,%
  morecomment=[l]\#,%
  morestring=[b]',%
  morestring=[b]",%
  stringstyle=\em,%
}

\lstdefinelanguage{testcase}{%
  moredelim=[is][\bfseries]{`}{`},%
  backgroundcolor=\color{gray!20},%
}

\lstdefinelanguage{file}{%
  frame=single,%
  backgroundcolor=\color{white},%
}

\lstset{language=py}
\lstset{basicstyle=\ttfamily}
\lstset{columns=fixed}
\lstset{upquote=true}
\lstset{showstringspaces=false}
\lstset{rangeprefix=\#\ }
\lstset{includerangemarker=false}

\newlist{certamen}{enumerate}{1}
\setlist[certamen]{%
  label=\arabic*.,
  font=\LARGE\bfseries,%
  labelindent=-.5in,%
  leftmargin=0pt,%
  labelsep=1em%
}


\newcommand{\diapo}[1]{
  %\begin{wrapfigure}[11]{l}{.35\textwidth}
    \vspace{2ex}
    \fbox{\includeslide[width=.3\textwidth]{#1}}
  %\end{wrapfigure}
}


\title{Módulos, funciones avanzadas}
\author{Programación \\ \url{http://progra.usm.cl}}
\date{}

\begin{document}
  \maketitle

  \section*{Objetivos de la clase}
  \begin{itemize}
    \item Introducir el concepto y el uso de los módulos.
    \item Enseñar a crear un módulo.
    \item Explicar algunos usos más avanzados de las funciones.
  \end{itemize}

  \section*{Diapositivas}

  \diapo{uso-modulos}

  Un \emph{módulo} en Python
  es un archivo que contiene definiciones
  que pueden ser importadas desde un programa para ser usadas.

  Python provee una vasta biblioteca de módulos
  que están disponibles para ser usados desde cualquier programa.
  Además, es posible instalar módulos externos
  para propósitos específicos.

  En esta diapositiva
  se muestran las dos maneras principales de usar un módulo.
  Además, se introduce dos de los módulos estándares de Python: \li!math! y \li!random!,
  que proveen respectivamente funciones matemáticas y funciones para generar números aleatorios.

  La primera manera de usar un módulo
  es importar nombres específicos desde él
  usando la sentencia
  \begin{lstlisting}
    from modulo import nombres
  \end{lstlisting}
  Los nombres van separados por comas.
  Los alumnos ya deberían estar familiarizados con esta forma,
  pues aparece usada varias veces en los apuntes de la primera unidad.

  La segunda manera es importar el módulo completo,
  y luego acceder a sus miembros usando el punto:
  \li!modulo.objeto!.
  Esta forma no ensucia el espacio de nombres global,
  pues todos los nombres quedan encapsulados en el módulo.

  Existe una tercera manera que es importar todos los nombres de un módulo
  usando la sentencia:
  \begin{lstlisting}
    from modulo import *
  \end{lstlisting}
  Sin embargo, esta forma no es recomendada,
  pues puede importar nombres que sobreescriban a otros que ya existan.
  Por lo mismo, esta manera no hay que enseñarla.

  Las funciones y constantes usadas en este ejemplo son:
  \begin{itemize}
    \item \li!math.exp(x)!: la exponencial \(e^x\);
    \item \li!math.cos(x)!: el coseno \(\cos x\);
    \item \li!math.pi!: la constante \(\pi\);
    \item \li!random.randrange(a, b, c)!: entrega un entero aleatorio
      en el rango definido por los valores \(a\), \(b\) y \(c\)
      (en el ejemplo, un entero aleatorio entre 0 y 9, inclusive);
    \item \li!random.choice(opciones)!: entrega uno de los valores de la lista de opciones
      (las listas se pasarán más adelante,
      pero por ahora basta saber que se delimitan con paréntesis cuadrados).
  \end{itemize}

  Muestre en la consola cómo al llamar las funciones \li!randrange! y \li!choice!
  siempre se obtiene un resultado distinto.

  \diapo{creacion-modulos}

  Además de usar los módulos estándares de Python,
  uno puede crear sus propios módulos con definiciones
  para ser reutilizadas en varios programas,
  o para organizar mejor una aplicación relativamente grande.

  La manera más simple de crear un módulo
  es escribir un archivo con extensión \verb!.py!
  y dejarlo en la misma carpeta que el programa que lo usa.
  En esta diapositiva se muestra un ejemplo:
  el archivo \verb!programa.py! es un programa que muestra el área y el perimetro de un círculo,
  usando funciones importadas desde un módulo llamado \verb!circulos!
  que está definido a continuación.

  El módulo define tres nombres (\li!pi!, \li!area! y \li!perimetro!),
  de los cuales dos son importados desde \verb!programa.py!.

  Enfatice que este módulo no sólo puede ser usado desde este programa,
  sino desde cualquier otro que necesite estas funciones.

  \diapo{programa-modulo}

  Un archivo con extensión \verb!.py! puede ser un programa o un módulo.
  Un programa a su vez puede tener sus propias definiciones de funciones
  que podrían ser útiles en otro programa,
  por lo que es deseable que puedan ser importadas.
  El problema de importar objetos desde un programa
  es que, al hacerlo, el programa también será ejecutado.

  El truco para evitar que esto ocurra es utilizar el valor
  de una variable llamada \li!__name__!
  que está definida automáticamente al ejecutar cualquier programa o módulo.
  Cuando el código es importado como un módulo,
  el valor de esta variable es el nombre del módulo.
  Cuando es ejecutado como un programa,
  su valor es el string \li!'__main__'!.
  Por lo tanto,
  para marcar una sección de código que no debe ser ejecutada
  al importar el archivo,
  hay que ponerla dentro de un \li!if __name__ == '__main__'!.

  En el ejemplo de la diapositiva
  se presenta el mismo programa del primer ejercicio de la clase pasada,
  salvo que el cuerpo del programa ha sido puesto dentro del \li!if!.
  De esta manera, las funciones \li!factorial! y \li!coef_bin!
  pueden ser importadas desde otros programas.

  \diapo{mult-retornos}

  Aquí comenzamos con los usos avanzados de funciones.

  Una función puede retornar más de un valor.
  Los valores de retorno deben ir separados por comas
  en la sentencia \li!return!,
  y deben ser recibidos usando varias variables,
  como se muestra en el ejemplo en la consola.

  Gracias a esta funcionalidad,
  en Python no es necesario usar variables globales o parámetros por referencia
  para que la función entregue más de un resultado,
  que es lo que suele hacerse en otros lenguajes.

  Técnicamente,
  lo que la función está retornando es una tupla de valores.
  Esto es parte de la materia de la próxima semana.

  \diapo{sin-retorno}

  En general, una función no debería tener efectos laterales,
  como modificar variables globales o mostrar cosas por pantalla.
  Toda la información que sale de la función
  debería estar contenida en los valores de retorno.

  Sin embargo,
  a veces es deseable usar las funciones
  como una manera de agrupar sentencias
  que interesan sólo por sus efectos secundarios.
  En este caso, la función puede no tener un valor de retorno.

  En el ejemplo mostrado en la diapositiva,
  se define una función con el único propósito
  de mostrar los datos que recibe en un formato más legible.

  Técnicamente, todas las funciones en Python tienen un valor de retorno.
  Si la función se termina sin haber encontrado una sentencia \li!return!,
  entonces el valor retornado será \li!None!.
  Esto no es necesario mencionarlo en clases.

  \diapo{parametros-por-omision}

  Una función puede tener parámetros opcionales.
  Una manera de hacerlo es asignando valores por omisión
  a algunos de sus parámetros.
  Cuando ellos no son pasados al momento de la llamada,
  toman el valor por omisión.

  En el ejemplo,
  la función \li!f! suma los valores de sus parámetros.
  Si sólo se pasa uno de ellos,
  entonces se toma como si el otro tuviera el valor 2.

  \diapo{func-parametros}

  En Python las funciones son valores como cualquier otro.
  Por ejemplo, se les puede crear un nuevo nombre
  con una simple asignación: \li!f = factorial!.
  Una de las ventajas que tiene esto
  es que es posible crear funciones que reciben a otras funciones como parámetro.

  En el ejemplo de la diapositiva,
  se define una función \li!sumar!
  que entrega la suma de todos los valores menores que \li!n!,
  pero antes aplicándoles una función,
  la que será referida con el nombre \li!f!.

  El programa llama a la función \li!sumar! tres veces,
  y en cada una de ellas pasa una función distinta como parámetro.
  De esta forma, se puede usar una única función
  para calcular la suma de los valores hasta 1000,
  la suma de los cuadrados hasta 1000
  y la suma de los cubos hasta 1000.

  \diapo{func-recursivas}

  Explique la definición de función recursiva:
  es una función que se llama a sí misma.

  Las funciones recursivas son útiles cuando un problema
  puede ser descompuesto de modo que aparezca una variación del mismo problema
  pero de un tamaño más pequeño.
  El ejemplo típico es el cálculo del factorial,
  que es presentado en la diapositiva.
  El factorial de \(n\) puede ser expresado en función del factorial de \(n - 1\).
  Explique en detalle esta descomposición.

  Toda función recursiva debe tener una condición de término
  para que su ejecución eventualmente termine.
  Típicamente, la mayoría de las funciones recursivas
  tienen un \li!if! que distingue entre el caso base
  (para el que el resultado es conocido)
  y el caso recursivo.
  En el programa del factorial,
  el caso base es \(0! = 1\).

  En lenguajes procedurales como Python,
  la recursión suele ser ineficiente en tiempo de ejecución,
  por lo que no se debe abusar de ella.
  Sin embargo, es útil para resolver ciertas clases de problemas
  de manera clara y elegante.
  El ejemplo clásico es el problema de las \href{http://en.wikipedia.org/wiki/Tower_of_Hanoi}{torres de Hanoi}.
  Más adelante veremos ejercicios que conviene resolver usando recursión.


\end{document}

