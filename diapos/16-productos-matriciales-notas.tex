\documentclass[10pt]{article}
\usepackage{beamerarticle}
\usepackage[spanish]{babel}
\usepackage[utf8]{inputenc}
\usepackage{fullpage}
\usepackage{xcolor}
\usepackage{listings}
\usepackage{textcomp}
\usepackage{mathpazo}
\usepackage{courier}
\usepackage{fancyvrb}
\usepackage{amsmath}
\usepackage{url}
\usepackage{hyperref}
\usepackage{pgfpages}
\usepackage{wrapfig}
\hyphenation{}
\usepackage{enumitem}

\setjobnamebeamerversion{16-productos-matriciales-diapos}

\newcommand{\onelinerule}{\rule[2.3ex]{0pt}{0pt}}
\newcommand{\twolinerule}{\rule[6.2ex]{0pt}{0pt}}
\newcommand{\respuesta}{\framebox[\textwidth]{\twolinerule}}
\newcommand{\nombre}{%
  \begin{tikzpicture}[xscale=.4,yscale=.7]
    \draw (0, 0) rectangle (22, 1);
  \end{tikzpicture}%
}
%\newcommand{\rol}   {\framebox[0.3\textwidth]{\onelinerule}}
\newcommand{\rol}{%
  \begin{tikzpicture}[xscale=.4,yscale=.7]
    \draw[gray!40] ( 0, 0) grid      ( 9, 1);
    \draw          ( 0, 0) rectangle ( 9, 1);
    \draw          (10, 0) rectangle (11, 1);
    \draw (9 + .2, .5) -- (10 - .2, .5);
  \end{tikzpicture}%
}
\newcommand{\li}{\lstinline}
\newcommand{\pond}[1]{[{\small\textbf{#1\%}}]}

\lstdefinelanguage{py}{%
  classoffset=0,%
    morekeywords={%
      False,class,finally,is,return,None,continue,for,lambda,try,%
      True,def,from,nonlocal,while,and,del,global,not,with,print,%
      as,elif,if,or,yield,assert,else,import,pass,break,except,in,raise},%
    keywordstyle=\color{black!80}\bfseries,%
  classoffset=1,
    morekeywords={int,float,str,abs,len,raw_input,exit,range,min,max,%
      set,dict,tuple,list,bool,complex,round,sum,all,any,zip,map,filter,%
      sorted,reversed,dir,file,frozenset,open,%
      array,zeros,ones,arange,linspace,eye,diag,dot},
    keywordstyle=\color{black!50}\bfseries,%
  classoffset=0,%
  sensitive=true,%
  morecomment=[l]\#,%
  morestring=[b]',%
  morestring=[b]",%
  stringstyle=\em,%
}

\lstdefinelanguage{testcase}{%
  moredelim=[is][\bfseries]{`}{`},%
  backgroundcolor=\color{gray!20},%
}

\lstdefinelanguage{file}{%
  frame=single,%
  backgroundcolor=\color{white},%
}

\lstset{language=py}
\lstset{basicstyle=\ttfamily}
\lstset{columns=fixed}
\lstset{upquote=true}
\lstset{showstringspaces=false}
\lstset{rangeprefix=\#\ }
\lstset{includerangemarker=false}

\newlist{certamen}{enumerate}{1}
\setlist[certamen]{%
  label=\arabic*.,
  font=\LARGE\bfseries,%
  labelindent=-.5in,%
  leftmargin=0pt,%
  labelsep=1em%
}


\newcommand{\diapo}[1]{
  %\begin{wrapfigure}[11]{l}{.35\textwidth}
    \vspace{2ex}
    \fbox{\includeslide[width=.3\textwidth]{#1}}
  %\end{wrapfigure}
}


\title{Productos matriciales}
\author{Programación \\ \url{http://progra.usm.cl}}
\date{}

\begin{document}
  \maketitle

  \section*{Objetivos de la clase}
  \begin{itemize}
    \item Enseñar los productos
      vector-vector, matriz-vector y matriz-matriz.
    \item Enseñar cómo obtener estos productos usando NumPy.
    \item Mostrar algunas aplicaciones de estos productos.
  \end{itemize}

  \section*{Diapositivas}

  \diapo{repaso-arreglos}

  Antes de comenzar con la materia de la clase de hoy,
  haga un breve repaso sobre la materia de arreglos,
  recordando lo siguiente.

  Para poder usar arreglos,
  hay que instalar el módulo NumPy,
  e importar sus funciones al principio del programa
  o de la sesión de consola
  usando la sentencia al principio de la diapositiva.

  La función \li!array! sirve para crear un arreglo
  con los valores indicados.
  En el primer ejemplo,
  se crean dos arreglos \li!a! y \li!b! de una dimensión y de tamaño 3.
  Las operaciones sobre arreglos se aplican elemento a elemento,
  y los operandos pueden ser arreglo-escalar o arreglo-arreglo.
  En el ejemplo,
  \li!c! es el resultado de multiplicar por 2 todos los elementos de \li!a!,
  y \li!d! es la suma elemento a elemento de \li!a! y \li!b!.
  Los arreglos de una dimensión también se llaman \emph{vectores}.

  Los arreglos de dos dimensiones
  también son creados usando la función \li!array!,
  pero pasando una lista de filas como argumento.
  Las operaciones también se hacen elemento a elemento.
  En el ejemplo,
  \li!g! es el resultado de sustraer 4 a todos los elementos de \li!e!,
  y  \li!h! tiene los elementos de \li!e!
  elevados al elemento correspondiente de \li!f!.
  Para operar sobre dos arreglos bidimensionales,
  ambos deben tener la misma forma (cantidades de filas y columnas).
  Los arreglos de dos dimensiones también se llaman \emph{matrices}.

  En la materia de esta clase,
  enseñaremos nuevas operaciones entre arreglos:
  el producto interno, el producto matriz-vector y el producto matriz-matriz.
  Estas operaciones se pueden implementar usando las operaciones elementales de arreglos,
  pero son tan usadas en la práctica
  que se justifica que estén ya implementadas eficientemente.
  Para todas estas operaciones, se usa la función \li!dot!.

  \diapo{producto-interno}

  El \emph{producto interno} (o \emph{producto punto}, o \emph{producto escalar})
  de dos vectores del mismo tamaño
  se calcula multiplicando los vectores elemento a elemento,
  y sumando los resultados.

  Muestre en el diagrama
  que \li!a! y \li!b! representan vectores de números reales de largo 5.
  Los resultados de los productos término a término
  son mostrados en gris, y el resultado final es mostrado más abajo.

  Puede preguntar a los estudiantes
  cómo implementar esta operación
  usando la materia vista hasta ahora.
  La solución es \li!sum(a * b)!.

  \diapo{producto-interno-numpy}

  Además de la solución \li!sum(a * b)!,
  el producto interno puede ser calculado
  usando la función \li!dot!.
  La sesión de consola de ejemplo
  muestra los mismos vectores del diagrama anterior,
  e ilustra que el resultado de la llamada a \li!dot!
  es el mismo que se obtiene usando \li!sum(a * b)!.

  Puede mostrar algunos ejemplos de producto interno
  en la consola para dejar más claro qué es lo que hace la operación.

  \diapo{ejercicios-producto-interno}

  Éstos son ejemplos sencillos
  para verificar que los alumnos entendieron el concepto.
  Pídales que digan los resultados de las operaciones.
  Las soluciones son, respectivamente, 27 y 8.

  \diapo{ejemplos-producto-interno}

  En esta diapositiva se presenta algunos ejemplos prácticos
  de uso del producto interno,
  para justificar mejor por qué existe una función especial
  para esta operación.

  En el primer ejemplo,
  se tiene un vector con los precios unitarios
  de algunos productos.
  Para obtener el valor total a pagar por una compra,
  hay que obtener su producto interno
  con el vector que tiene las cantidades que se comprará
  de cada producto.
  La asociación entre cada precio y cada cantidad
  está dada por sus posiciones en el arreglo.
  Cada índice corresponde a un producto.

  En el segundo ejemplo,
  se muestra cómo calcular un promedio de notas
  en el que cada nota tiene distinta ponderación.
  El arreglo en la primera expresión
  tiene los porcentajes asignados para cada nota,
  y luego es dividido por 100 para normalizarlo
  (en general, para calcular promedios ponderados
  debe cumplirse que \li!sum(ponderaciones) == 1!).
  Teniendo las notas y las ponderaciones,
  el promedio se obtiene como el producto interno
  de ambos arreglos.

  El tercer ejemplo
  viene del ejercicio presentado en la clase de la semana pasada
  que consistía en calcular el área de unos rectángulos
  bajo la curva de la exponencial.
  Puede hacer un diagrama en la pizarra para recordarlo.
  Primero, se crea un arreglo \li!x!
  que tiene 101 puntos equiespaciados entre 0 y 5
  (es decir, hay 99 puntos intermedios entre ambos extremos,
  y por lo tanto hay 100 subintervalos).
  Luego,
  se obtiene las alturas de cada rectángulo
  evaluando todos los puntos excepto el último en la función \(e^x\)
  (ya que la función es creciente
  y por lo tanto es el lado izquierdo del rectángulo
  el que toca la curva).
  Los largos de las bases son todos iguales,
  y son el largo del intervalo (5) dividido por la cantidad de subintervalos;
  aquí se usa la función \li!ones! para crear un arreglo de valores iguales.
  Finalmente, el área total de los rectángulos
  es el producto interno entre los vectores de alturas y bases.

  \diapo{producto-matriz-vector}

  La segunda operación de la clase
  es el producto matriz-vector.
  Posiblemente haya algunos alumnos que ya conozcan esta operación,
  pero lo más seguro es que la mayoría aún no la haya visto.
  El producto matriz-vector
  consiste en calcular los productos internos del vector
  con todas las \emph{filas} de la matriz.

  Discuta con los alumnos:
  ¿qué condiciones deben satisfacer la forma de la matriz
  y el largo del vector para que la operación pueda ser realizada?,
  ¿qué tamaño tendrá el resultado?
  Respuestas: el largo del vector
  debe ser igual al número de columnas de la matriz,
  y el vector resultado tiene tamaño igual
  al número de filas.

  \diapo{producto-matriz-vector-numpy}

  En NumPy, el producto matriz-vector
  también se obtiene usando la función \li!dot!.
  Para que la operación funcione de la manera presentada
  en la diapositiva anterior,
  el primer argumento debe ser la matriz
  y el segundo debe ser el vector.
  Si se ponen en el orden incorrecto,
  se obtendrá otro resultado
  (a saber, los productos internos
  entre \li!x! y las columnas de \li!a!).

  \diapo{ejercicios-producto-matriz-vector}

  Ejemplos sencillos
  para verificar que los alumnos entendieron el concepto.
  Pídales que digan los resultados de las operaciones.
  Las soluciones son, respectivamente,
  \li!array([49, 31])! (los mismos valores en orden inverso)
  y \li!array([8, 2])!.

  \diapo{ejercicio-nutrientes}
  \diapo{caso-nutrientes}

  (Este ejercicio está sacado del libro \emph{Linear Algebra and Its Applications}, de David~C.~Lay).

  Lea en voz alta el ejercicio, explique el caso de prueba
  y discuta con los alumnos las estrategias posibles
  para resolver el problema.

  La solución es simplemente el producto matriz-vector
  entre la tabla de aportes por alimento
  y el vector con las cantidades de cada elemento.
  Intente que los estudiantes se den cuenta de esto por sí mismos.

  El vector de cantidades debe ser ingresado por el usuario,
  mientras que la tabla es constante y está incluida en el código.
  Para ingresar las cantidades,
  puede inicializarse un vector de ceros
  y luego asignar sus elementos uno a uno,
  o bien agregar las cantidades a una lista
  y al final convertirla a un arreglo.

  El código de una solución posible
  está incluido en los programas adjuntos a la clase.
  En clases, puede desarrollarla de otra manera,
  o guiarse según las sugerencias de los alumnos.

  \diapo{producto-matriz-matriz}

  La tercera operación de la clase
  es la multiplicación matricial.

  Muestre en el diagrama
  que el producto de dos matrices
  se obtiene como los productos internos
  de las filas de la primera matriz
  y las columnas de la segunda.
  Es casi lo mismo que el producto matriz-vector,
  pero poniendo varios vectores pegados uno junto al otro.

  Si le queda tiempo en la clase,
  discuta que condiciones deben satisfacer las matrices,
  y qué forma tiene el resultado.
  (El número de columnas de la primera debe coincidir
  con el número de filas de la segunda,
  y el resultado tiene tantas filas como la primera
  y tantas columnas como la segunda).

  \diapo{producto-matriz-matriz-numpy}

  El producto matriz-matriz también se obtiene con la función \li!dot!.
  Al igual que para la operación anterior,
  el orden de los operandos es importante.

  \diapo{ejercicio-acero}

  (Ejercicio sacado de \href{http://optimierung.mathematik.uni-kl.de/mamaeusch/veroeffentlichungen/ver_texte/matrizenrechnung_spanish.pdf}{este apunte}).

  Si queda tiempo suficiente,
  lea el enunciado del ejercicio,
  y discuta con los estudiantes
  la estrategia para resolverlo.
  La solución consiste en multiplicar matricialmente ambas tablas,
  y luego obtener los totales por columna.
  La empresa que entrega el mayor beneficio
  es la que está asociada a la columna con el menor total.

  Discuta la interpretación del resultado del producto matricial.
  Cada elemento de la matriz producto
  indica cuánto se debería pagar por las materias primas
  a una empresa dada en una semana dada.
  Todos estos valores incluyen tanto el hierro como el carbón.
  La matriz que se obtiene es de \(3\times 3\).

\end{document}

