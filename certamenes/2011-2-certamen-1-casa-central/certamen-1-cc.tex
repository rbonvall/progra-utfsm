\documentclass[11pt,spanish]{article}
\usepackage[utf8]{inputenc}
\usepackage{babel}
\usepackage{fullpage}
\usepackage{listings}
\usepackage{mathpazo}
\usepackage{enumitem}
\usepackage{courier}
\usepackage{xcolor}
\usepackage{textcomp}
\usepackage{amsmath}
\usepackage{amssymb}
\usepackage{tikz}
\usepackage{fancyhdr}
\usepackage{graphics}
\usepackage{array}

\newcommand{\titulo}{Certamen 1, ¿?}
\newcommand{\cc}[1]{\hfil\texttt{#1}\hfil}
\newcommand{\pond}[1]{[{\small\textbf{#1\%}}]}

\pagestyle{fancy}
\lhead{%
  {\Large\bfseries Programación---\titulo} \\
  Nombre: \nombre\hfill
  Rol:    \rol
  \vspace{2ex}
}
\chead{}\rhead{}\lfoot{}\cfoot{}\rfoot{}
\renewcommand{\headrulewidth}{0pt}
\addtolength{\headheight}{7ex}
\headsep=4ex


\newcommand{\onelinerule}{\rule[2.3ex]{0pt}{0pt}}
\newcommand{\twolinerule}{\rule[6.2ex]{0pt}{0pt}}
\newcommand{\respuesta}{\framebox[\textwidth]{\twolinerule}}
\newcommand{\nombre}{%
  \begin{tikzpicture}[xscale=.4,yscale=.7]
    \draw (0, 0) rectangle (22, 1);
  \end{tikzpicture}%
}
%\newcommand{\rol}   {\framebox[0.3\textwidth]{\onelinerule}}
\newcommand{\rol}{%
  \begin{tikzpicture}[xscale=.4,yscale=.7]
    \draw[gray!40] ( 0, 0) grid      ( 9, 1);
    \draw          ( 0, 0) rectangle ( 9, 1);
    \draw          (10, 0) rectangle (11, 1);
    \draw (9 + .2, .5) -- (10 - .2, .5);
  \end{tikzpicture}%
}
\newcommand{\li}{\lstinline}
\newcommand{\pond}[1]{[{\small\textbf{#1\%}}]}

\lstdefinelanguage{py}{%
  classoffset=0,%
    morekeywords={%
      False,class,finally,is,return,None,continue,for,lambda,try,%
      True,def,from,nonlocal,while,and,del,global,not,with,print,%
      as,elif,if,or,yield,assert,else,import,pass,break,except,in,raise},%
    keywordstyle=\color{black!80}\bfseries,%
  classoffset=1,
    morekeywords={int,float,str,abs,len,raw_input,exit,range,min,max,%
      set,dict,tuple,list,bool,complex,round,sum,all,any,zip,map,filter,%
      sorted,reversed,dir,file,frozenset,open,%
      array,zeros,ones,arange,linspace,eye,diag,dot},
    keywordstyle=\color{black!50}\bfseries,%
  classoffset=0,%
  sensitive=true,%
  morecomment=[l]\#,%
  morestring=[b]',%
  morestring=[b]",%
  stringstyle=\em,%
}

\lstdefinelanguage{testcase}{%
  moredelim=[is][\bfseries]{`}{`},%
  backgroundcolor=\color{gray!20},%
}

\lstdefinelanguage{file}{%
  frame=single,%
  backgroundcolor=\color{white},%
}

\lstset{language=py}
\lstset{basicstyle=\ttfamily}
\lstset{columns=fixed}
\lstset{upquote=true}
\lstset{showstringspaces=false}
\lstset{rangeprefix=\#\ }
\lstset{includerangemarker=false}

\newlist{certamen}{enumerate}{1}
\setlist[certamen]{%
  label=\arabic*.,
  font=\LARGE\bfseries,%
  labelindent=-.5in,%
  leftmargin=0pt,%
  labelsep=1em%
}



\begin{document}

  \begin{enumerate}[font=\Large\bfseries]

    % Entender programas
    \item%[1a.]
      \pond{25}
      Indique qué es lo que imprimen los siguientes programas.

      \foreach \x in {1,2} {
        \noindent
        \begin{minipage}[b]{.5\textwidth}
          \lstinputlisting{p\x.py}
          \framebox[.8\textwidth]{\rule[10ex]{0pt}{0pt}}
          \vspace{0.4em}
        \end{minipage}
      }

    %\item[1b.]
      Rutee el siguiente programa
      e indique qué es lo que imprime.

      Cada vez que el valor de una variable cambie,
      ponga su valor en una nueva fila de la tabla.
      La tabla tiene filas de sobra.

      \begin{minipage}[T]{.5\textwidth}
        \lstinputlisting{ruteo.py}
        \framebox[.8\textwidth]{\rule[10ex]{0pt}{0pt}}
      \end{minipage}
      \begin{minipage}[t]{.4\textwidth}\centering
        \begin{tabular}{|*{5}{p{2.6em}|}}\hline
            \cc{a} & \cc{c} & \cc{x} & \cc{y} & \cc{z} \\ \hline\hline
            &&&& \\\hline &&&& \\\hline &&&& \\\hline &&&& \\\hline &&&& \\\hline
            &&&& \\\hline &&&& \\\hline &&&& \\\hline &&&& \\\hline &&&& \\\hline
            &&&& \\\hline &&&& \\\hline &&&& \\\hline &&&& \\\hline &&&& \\\hline
            &&&& \\\hline &&&& \\\hline &&&& \\\hline &&&& \\\hline &&&& \\\hline
            &&&& \\\hline &&&& \\\hline &&&& \\\hline &&&& \\\hline &&&& \\\hline
         \end{tabular}
      \end{minipage}

    \newpage
    \item
      \pond{25}
      Un \emph{polinomio} de grado \(n\)
      es una función matemática que tiene la forma:
      \[
        p(x) =
        a_0     +
        a_1 x   +
        a_2 x^2 +
        a_3 x^3 +
        \cdots +
        a_n x^n.
      \]
      Los valores \(a_0, \ldots, a_n\)
      son los \emph{coeficientes} del polinomio,
      y \(x\) es la \emph{variable independiente}.

      \begin{minipage}[t]{.63\textwidth}
        Desarrolle un programa
        que evalúe un polinomio.
        \\[1ex]
        Primero,
        el usuario debe ingresar \(x\).
        A continuación,
        debe ingresar los coeficientes en orden.
        Para indicar que todos los coeficientes han sido ingresados,
        se debe escribir el texto \li!FIN!.
        Finalmente,
        el programa debe mostrar
        el valor de \(p(x)\).
        \\[1ex]
        El ejemplo de la derecha
        muestra cómo evaluar el polinomio
        \(p(x) = -7 - 3x^2 + 2x^3\) en \(x = 2.1\).
      \end{minipage}
      \hfill
      \begin{minipage}[t]{.26\textwidth}
        \lstinputlisting[language=testcase,frame=single]{caso-polinomio.txt}
      \end{minipage}

    \newpage
    \item
      \pond{25}
      El \emph{intercalao} es un juego muy popular
      entre los niños de la aldea de Pythópolis.

      El juego consiste en lanzar varias veces una moneda.
      En cada lanzamiento,
      el resultado puede ser cara (\li!C!) o sello (\li!S!).

      Un jugador gana cuando durante cuatro lanzamientos consecutivos
      aparecen caras y sellos intercalados
      (es decir, ningún resultado aparece dos veces seguidas),
      y pierde cuando un mismo resultado aparece
      cuatro veces seguidas.

      Escriba un programa que reciba como entrada
      los resultados de los lanzamientos
      hasta que termine el juego,
      y le indique al usuario si ganó o perdió.

      \begin{minipage}[t]{.26\textwidth}
        \lstinputlisting[language=testcase,frame=single,linerange=CASO\ 1-FIN\ CASO\ 1]{casos-intercalao.txt}
      \end{minipage}
      \hspace{1em}
      \begin{minipage}[t]{.26\textwidth}
        \lstinputlisting[language=testcase,frame=single,linerange=CASO\ 2-FIN\ CASO\ 2]{casos-intercalao.txt}
      \end{minipage}

    \newpage
    \item
      \pond{25}

  \end{enumerate}
\end{document}

