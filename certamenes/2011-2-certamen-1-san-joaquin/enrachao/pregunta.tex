El \emph{enrachao} es un juego muy popular
entre los niños de la aldea de Pythonia.

El juego consiste en lanzar varias veces un dado.
Apenas aparece el número 1, el jugador pierde.
Para ganar, a un jugador debe aparecerle un número
(distinto de 1)
tantas veces consecutivas como indica el mismo número
(por ejemplo, el 5 cinco veces seguidas).

Escriba un programa que reciba como entrada
todos los números obtenidos al lanzar el dado
hasta que termine el juego,
y le indique al usuario si ganó o perdió.

\begin{minipage}[t]{.26\textwidth}
  \lstinputlisting[language=testcase,frame=single,linerange=CASO\ 1-FIN\ CASO\ 1]{enrachao/casos-enrachao.txt}
\end{minipage}
\hspace{1em}
\begin{minipage}[t]{.26\textwidth}
  \lstinputlisting[language=testcase,frame=single,linerange=CASO\ 2-FIN\ CASO\ 2]{enrachao/casos-enrachao.txt}
\end{minipage}


