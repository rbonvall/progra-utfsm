La banda de rock Strip'n Split
ha concluído su gira mundial.
La información de cada uno de sus conciertos
está almacenada en una lista
en orden cronológico.

\begin{minipage}[t]{.45\textwidth}
  Cada elemento de la lista
  es un diccionario con tres llaves:
  \li!'ciudad'!, \li!'publico'! y \li!'canciones'!.
  El valor asociado a \li!'publico'!
  es la cantidad de personas que asistió al concierto,
  y el asociado a \li!'canciones'!
  es la lista de las canciones que fueron tocadas
  en ese concierto.

  \begin{enumerate}
    \item
      Escriba una función llamada
      \li!cuantos_escucharon(c)!,
      que retorne la cantidad de personas
      que escucharon la canción \li!c!.
    \item
      Escriba una función llamada
      \li!mismo_concierto(c1, c2)!,
      que retorne \li!True! o \li!False!
      para indicar si las canciones \li!c1! y \li!c2!
      fueron tocadas alguna vez en el mismo concierto.
  \end{enumerate}
\end{minipage}
\hspace{2em}
\begin{minipage}[t]{.55\textwidth}
  \small
  \lstinputlisting[linerange=CONCIERTOS-FIN\ CONCIERTOS]
     {strip-n-split/pauta3-4.py}
\end{minipage}

