Una \emph{página web}
es simplemente un archivo de texto
que describe las distintas partes de un documento
usando un lenguaje llamado HTML.

\begin{minipage}[t]{0.7\textwidth}
  \parskip=2ex
  Las \emph{tablas} son uno de los componentes
  que pueden ser parte de una página web.
  En un archivo HTML,
  el comienzo de una tabla

  Una manera simple de crear una tabla
  guardar una

  \begin{enumerate}[leftmargin=0pt,label=\emph{\alph*})]

    \item
      Escriba la función \li!convertir_csv_a_html(tabla)!
      que convierta un archivo CSV a otro HTML.
      El parámetro es el nombre del archivo sin la extensión.

      Por ejemplo,
      para crear el archivo \verb!frutas.html!
      a partir de los datos del archivo \verb!frutas.csv!
      habría que llamar así a la función:
      \lstinputlisting[linerange=CASO\ 1-FIN\ CASO\ 1]{tabla-html/programa.py}

      La función no debe retornar nada; sólo debe crear el archivo.

    \item
      Escriba la función \li!buscar_valor(tabla, valor)!
      que busque el \li!valor! en la \li!tabla! del archivo HTML,
      y retorne una tupla con la fila y la columna
      de la celda donde está el valor.

  \end{enumerate}

\end{minipage}
\hfill
\begin{minipage}[t]{0.25\textwidth}
  \centering
  Tabla de frutas: \\[1ex]
  \begin{tabular}{|l|r|r|}  \hline
    Manzana &  7 & \$615 \\ \hline
    Uva     & 10 & \$399 \\ \hline
    Kiwi    &  2 & \$471 \\ \hline
  \end{tabular}
  \vspace{1ex}

  Archivo \verb+frutas.html+:
  \lstinputlisting[language=html,frame=single]{tabla-html/frutas-ejemplo.html}

  Archivo \verb+frutas.csv+:
  \lstinputlisting[frame=single]{tabla-html/frutas.csv}
\end{minipage}

