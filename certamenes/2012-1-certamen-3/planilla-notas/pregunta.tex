Los profesores de la asignatura de Programología están hartos
de calcular manualmente las notas de cada certamen.
Para apurar un poco más el proceso
decidieron implementar un programa que los ayude en esta tarea.

Los certámenes consisten de varias preguntas.
Cada pregunta tiene un puntaje máximo posible y una ponderación.
La ponderación es el porcentaje de la nota final
que aporta cada pregunta.
La nota del certamen será un entero ente 0 y 100.

Para cada certamen,
se poblará una matriz en que cada columna está asociada a una pregunta
y cada fila contiene los puntajes de un alumno.
Además,
se creará arreglos con las ponderaciones y los puntajes máximos
de cada pregunta:
\lstinputlisting[linerange=PLANILLA-FIN\ PLANILLA]{planilla-notas/programa.py}


\begin{enumerate}[leftmargin=0pt,label=\emph{\alph*})]

  \item
    Escriba la función \li!calcular_notas(ponderaciones, totales, puntajes)!
    que retorne el arreglo con las notas de los estudiantes:
    \lstinputlisting[linerange=CASO\ 1-FIN\ CASO\ 1]{planilla-notas/programa.py}

\end{enumerate}

