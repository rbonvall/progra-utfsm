(Esta pregunta es la continuación de la anterior).

Escriba la función \li!combinar_mapas(ma, mb)!,
que reciba como parámetros dos mapas
como los descritos en la pregunta anterior,
y retorne un nuevo mapa que incluya
todas las ciudades y todos los caminos
de ambos mapas.

\begin{tikzpicture}[xscale=.75, yscale=.6]
  \newcommand\ciudad[3]{%
    \node[draw, circle] (#1) at (#2, #3) {#1};%
  }
  \newcommand\camino[2]{%
    \draw (#1) -- (#2);%
  }
  \def\ciudadesMapaUno{
    \ciudad{A}{0}{2}
    \ciudad{H}{1}{0}
  }
  \def\caminosMapaUno{
    \camino{A}{B}
    \camino{A}{H}
    \camino{D}{H}
    \camino{E}{H}
  }
  \def\ciudadesMapaDos{
    \ciudad{F}{5}{4}
    \ciudad{G}{5}{2}
  }
  \def\caminosMapaDos{
    \camino{B}{F}
    \camino{D}{E}
    \camino{D}{F}
    \camino{D}{G}
    \camino{F}{G}
  }
  \def\ciudadesMapaAmbos{
    \ciudad{B}{2}{4}
    \ciudad{D}{3}{2}
    \ciudad{E}{3}{0}
  }
  \def\caminosMapaAmbos{
    \camino{B}{D}
  }
  \def\yLey{-1.7}

  % mapa 1
  \begin{scope}
    \ciudadesMapaUno
    \ciudadesMapaAmbos
    \caminosMapaUno
    \caminosMapaAmbos
    \node[anchor=center] at (1.5, \yLey) {Mapa 1.};
  \end{scope}

  % mapa 2
  \begin{scope}[xshift=5cm]
    \ciudadesMapaDos
    \ciudadesMapaAmbos
    \caminosMapaDos
    \caminosMapaAmbos
    \node[anchor=center] at (3.0, \yLey) {Mapa 2.};
  \end{scope}

  % mapa combinado
  \begin{scope}[xshift=16cm]
    \ciudadesMapaUno
    \ciudadesMapaDos
    \ciudadesMapaAmbos
    \caminosMapaUno
    \caminosMapaDos
    \caminosMapaAmbos
    \node[anchor=center] at (3.0, \yLey) {Mapa combinado.};
  \end{scope}
\end{tikzpicture}

