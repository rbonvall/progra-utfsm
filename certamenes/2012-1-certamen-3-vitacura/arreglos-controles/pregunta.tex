El profesor Sansanales desea desarrollar un programa en Python
para ayudarse a calcular las notas finales a final de semestre.
La planilla de notas de controles de su curso
está representada como una matriz,
donde cada fila contiene las notas de un alumno
a lo largo del semestre:
\lstinputlisting[linerange=CONTROLES-FIN\ CONTROLES]{arreglos-controles/programa.py}

Los nombres de sus alumnos están almacenados
en una lista global llamada \li!alumnos!,
en el mismo orden en que están asociados a las filas de la matriz:
\lstinputlisting[linerange=ALUMNOS-FIN\ ALUMNOS]{arreglos-controles/programa.py}

Para cada control,
el profesor premia al alumno que obtuvo la nota más alta
otorgándole una estrellita.
Si varios alumnos obtuvieron la nota más alta,
todos ellos reciben estrellita.

Cuando termine el semestre,
el profesor decidirá un valor de \(n\),
y premiará con 10 puntos en la nota final
a los alumnos que hayan obtenido \(n\) o más estrellitas.

\begin{enumerate}[leftmargin=0pt,label=\emph{\alph*})]

  \item
    Escriba la función \li!contar_estrellitas(controles)!
    que a partir de la matriz de notas de controles
    retorne un arreglo con el total de estrellitas que obtuvo cada alumno:
    \lstinputlisting[linerange=CASO\ 1-FIN\ CASO\ 1]{arreglos-controles/programa.py}

  \item
    Escriba la función \li!notas_finales(controles, n)!
    que entregue el arreglo de notas finales,
    después de haber sumado los premios
    por tener más de \(n\) estrellitas:
    \lstinputlisting[linerange=CASO\ 2-FIN\ CASO\ 2]{arreglos-controles/programa.py}

  \item
    Escriba la función \li!alumno_con_mas_estrellitas(controles)!
    que retorne el nombre del alumno con más estrellitas:
    \lstinputlisting[linerange=CASO\ 3-FIN\ CASO\ 3]{arreglos-controles/programa.py}

\end{enumerate}

