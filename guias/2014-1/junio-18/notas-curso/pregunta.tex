\lstset{language=file,frame=single}

El archivo \texttt{curso.txt} contiene los datos
de los estudiantes de un curso.
Cada línea corres\-ponde a un estudiante,
y contiene varios datos separados por dos puntos («\verb+:+»)
como en el si\-guiente ejemplo:
\lstinputlisting[linerange=-5]{notas-curso/curso.txt}

Los datos son:
el nombre, el apellido, la fecha de nacimiento
y las cinco notas obtenidas en el curso.

Escriba el programa \texttt{pagina\_web.py}
que genere un archivo llamado \texttt{curso.html}
con un contenido como el siguiente:
\lstinputlisting[basicstyle=\small\ttfamily]{notas-curso/ejemplo.html}
(Por brevedad aquí sólo se muestra los primeros dos estu\-diantes;
usted debe agregarlos todos).
Los datos listados son:
el nombre completo,
las notas de los cinco certámenes
y el promedio final (encerrado entre \verb+<b>+ y \verb+</b>+).

El archivo generado es una página web.
Ábralo con su nave\-gador favorito
para comprobar que fue creado correctamente.
