El \emph{factorial} de un número natural \(n\)
es el producto de los enteros desde 1 hasta \(n\):
\[ n! = 1\cdot 2\cdot\,\cdots\, \cdot(n - 1)\cdot n. \]
Escriba un programa que calcule el factorial del número ingresado por el usuario:

\begin{minipage}[t]{.21\textwidth}
  \lstinputlisting[language=testcase,frame=single,linerange=CASO\ 1-FIN\ CASO\ 1]{factorial/casos.txt}
\end{minipage}
\quad
\begin{minipage}[t]{.21\textwidth}
  \lstinputlisting[language=testcase,frame=single,linerange=CASO\ 2-FIN\ CASO\ 2]{factorial/casos.txt}
\end{minipage}
\quad
\begin{minipage}[t]{.21\textwidth}
  \lstinputlisting[language=testcase,frame=single,linerange=CASO\ 3-FIN\ CASO\ 3]{factorial/casos.txt}
\end{minipage}
\quad
\begin{minipage}[t]{.21\textwidth}
  \lstinputlisting[language=testcase,frame=single,linerange=CASO\ 4-FIN\ CASO\ 4]{factorial/casos.txt}
\end{minipage}
