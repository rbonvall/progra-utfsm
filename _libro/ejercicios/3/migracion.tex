\section{Migración de poblaciones}

\emph{Ejercicio sacado de} {[}Lay97{]}\_.

Estudios demográficos muestran que, cada año, el 5\% de la población de
una ciudad se muda a los suburbios (y el 95\% se queda), mientras que el
3\% de la población de los suburbios se muda a la ciudad (y el 97\% se
muda).

Estos datos pueden ser representados en una \textbf{matriz de
migración}:

\[M =
\frac{1}{100}
\begin{bmatrix}
95 &  3 \\
5 & 97 \\
\end{bmatrix}\]

\begin{enumerate}
\item
  Escriba un programa que pregunte al usuario cuáles son las poblaciones
  de la ciudad y los suburbios en el año 2011, y entregue una tabla con
  las poblaciones proyectadas para los siguientes 10 años:
\item
  Considere ahora la siguiente variación. Suponga que todos los años hay
  14000 personas que se mudan a la ciudad desde fuera de la región (no
  desde los suburbios) y 9000 personas abandonan la región; además, hay
  13000 personas que se mudan anualmente a los suburbios desde fuera de
  la ciudad.

  Modifique el programa anterior para resolver este problema.
\end{enumerate}
