\section{Cuadrado mágico}

Un `cuadrado mágico`\_ es una disposición de números naturales en una
tabla cuadrada, de modo que las sumas de cada columna, de cada fila y de
cada diagonal son iguales.

Los cuadrados mágicos más populares son aquellos que tienen los números
consecutivos desde el 1 hasta `n\^{}2`, donde `n` es el número de filas
y de columnas del cuadrado.

Por ejemplo, el siguiente es un cuadrado mágico con `n = 4`. Todas sus
filas, columnas y diagonales suman 34:

\includegraphics{../../diagramas/cuadrado-magico.png}

\begin{enumerate}
\item
  Escriba una función que reciba un arreglo cuadrado de enteros de
  `ntimes n`, e indique si está conformado por los números consecutivos
  desde 1 hasta `n\^{}2`:

\begin{lstlisting}
>>> from numpy import array
>>> consecutivos(array([[3, 1, 5],
...                     [4, 7, 2],
...                     [9, 8, 6]]))
True
>>> consecutivos(array([[3, 1, 4],
...                     [4, 0, 2],
...                     [9, 9, 6]]))
False
\end{lstlisting}
\item
  Escriba una función que reciba un arreglo e indique si se trata o no
  de un cuadrado mágico:

\begin{lstlisting}
>>> es_magico(array([[3, 1, 5],
...                  [4, 7, 2],
...                  [9, 8, 6]]))
False
>>> es_magico(array([[2, 7, 6],
...                  [9, 5, 1],
...                  [4, 3, 8]]))
True
\end{lstlisting}
\end{enumerate}
