\section{Informe de producción de gas}

En un informe anual de SansanoGas S.A., el presidente informa a sus
accionistas la cantidad anual de producción de barriles de 50 litros de
lubricantes normal, extra y súper, en sus dos refinerías:

\ctable[pos = H, center, botcap]{llll}
{% notes
}
{% rows
\FL
Refinería & Normal & Extra & Súper
\ML
A & 3000 & 7000 & 2000
\\\noalign{\medskip}
B & 4000 & 500 & 600
\LL
}

Además, informa que en cada barril de 50 litros de lubricante existe la
siguiente composición en litros de aceites finos, alquitrán y grasas
residuales:

\ctable[pos = H, center, botcap]{llll}
{% notes
}
{% rows
\FL
Componente & Normal & Extra & Súper
\ML
Aceites finos & 10 & 5 & 35
\\\noalign{\medskip}
Alquitrán & 15 & 4 & 31
\\\noalign{\medskip}
Grasas residuales & 18 & 2 & 30
\LL
}

\begin{enumerate}
\item
  Escriba la función \lstinline!totales_anuales(a, b)! que reciba como
  parámetros ambas matrices y retorne un arreglo con los totales de
  aceites finos, alquitrán y grasas residuales presentes en la
  producción anual.
\item
  Escriba la función \lstinline!maximo_alquitran(a, b)! que reciba como
  parámetros ambas matrices y retorne el máximo de litros de alquitrán
  consumidos por ambas refinerías.
\item
  Determine cuál es la matriz que entrega el consumo de todos los
  elementos que forman parte de un lubricante, en cada refinería.
\end{enumerate}
