\section{Construcción de una dieta}

\emph{Ejercicio sacado de} {[}Lay97{]}\_.

La dieta Cambridge es una dieta que fue popular en la década de los 80,
y fue el resultado de más de ocho años de trabajo clínico e
investigación de un equipo de científicos liderados por el doctor Alan
H. Howard en la Universidad de Cambridge.

La dieta combina un balance preciso de carbohidratos, proteínas de alta
calidad y grasa, junto con vitaminas, minerales, oligoelementos y
electrolitos. Millones de personas han usado la dieta en años recientes
para bajar rápidamente de peso.

Para alcanzar las proporciones de nutrientes deseadas, el doctor Howard
debió incorporar una gran variedad de comidas en la dieta. Cada comida
provee varios de los nutrientes, pero no en las proporciones correctas.
Por ejemplo, la leche descremada es una buena fuente de proteínas, pero
contiene mucho calcio. Por esto, se usó harina de soya (que tiene poco
calcio) para proveer las proteínas; sin embargo, tiene proporcionalmente
mucha grasa, por lo que se agregó suero de leche a la dieta, que
desafortunadamente contiene muchos carbohidratos\ldots{} como se hace
evidente, el delicado problema de balancear los nutrientes es complejo.

La siguiente tabla muestra el aporte en nutrientes por cada 100 gramos
de cada uno de los tres ingredientes (leche descremada, harina de soya y
suero de leche):

\ctable[pos = H, center, botcap]{llll}
{% notes
}
{% rows
\FL
Nutrientes & LD & HS & SL
\ML
Proteínas & 36 & 51 & 13
\\\noalign{\medskip}
Carbohidratos & 52 & 34 & 74
\\\noalign{\medskip}
Grasas & 0 & 7 & 1.1
\LL
}

La dieta de Cambridge debe proveer 33 gramos de proteínas, 45 gramos de
carbohidratos y 3 gramos de grasa.

Escriba un programa que muestre qué cantidades de ingredientes se debe
usar para satisfacer los requerimientos de la dieta de Cambridge.
