\section{Sudoku}

\begin{figure}
  \centering
  \section{Sudoku}

\begin{figure}
  \centering
  \section{Sudoku}

\begin{figure}
  \centering
  \section{Sudoku}

\begin{figure}
  \centering
  \input{diagramas/sudoku}
  \caption{%
    A la izquierda: sudoku sin resolver.
    A la derecha: sudoku resuelto.
  }
  \label{fig:sudoku}
\end{figure}

El \emph{sudoku} es un puzzle que consiste en llenar
una grilla de \(9 × 9\)
(como se ve en la figura~\ref{fig:sudoku})
con los dígitos del 1 al 9,
de modo que no haya ningún valor repetido en
cada fila, en cada columna y en cada uno de las regiones de \(3 × 3\)
marcadas por las líneas más gruesas.

El sudoku sin resolver tiene algunos de los dígitos puestos de antemano
en la grilla. Cuando el puzzle ha sido resuelto, todas las casillas
tienen un dígito, y entre todos satisfacen las condiciones señaladas.

En un programa, un sudoku resuelto puede ser guardado en un arreglo de
\(9 × 9\):

\begin{lstlisting}
from numpy import array
sr = array([[4, 2, 6, 5, 7, 1, 3, 9, 8],
            [8, 5, 7, 2, 9, 3, 1, 4, 6],
            [1, 3, 9, 4, 6, 8, 2, 7, 5],
            [9, 7, 1, 3, 8, 5, 6, 2, 4],
            [5, 4, 3, 7, 2, 6, 8, 1, 9],
            [6, 8, 2, 1, 4, 9, 7, 5, 3],
            [7, 9, 4, 6, 3, 2, 5, 8, 1],
            [2, 6, 5, 8, 1, 4, 9, 3, 7],
            [3, 1, 8, 9, 5, 7, 4, 6, 2]])
\end{lstlisting}

Escriba la función \lstinline!solucion_es_correcta(sudoku)! que reciba
como parámetro un arreglo de \(9 × 9\) representando un sudoku resuelto, y
que indique si la solución es correcta (es decir, si no hay elementos
repetidos en filas, columnas y regiones):

\begin{lstlisting}
>>> solucion_es_correcta(s)
True
>>> s[0, 0] = 9
>>> solucion_es_correcta(s)
False
\end{lstlisting}


  \caption{%
    A la izquierda: sudoku sin resolver.
    A la derecha: sudoku resuelto.
  }
  \label{fig:sudoku}
\end{figure}

El \emph{sudoku} es un puzzle que consiste en llenar
una grilla de \(9 × 9\)
(como se ve en la figura~\ref{fig:sudoku})
con los dígitos del 1 al 9,
de modo que no haya ningún valor repetido en
cada fila, en cada columna y en cada uno de las regiones de \(3 × 3\)
marcadas por las líneas más gruesas.

El sudoku sin resolver tiene algunos de los dígitos puestos de antemano
en la grilla. Cuando el puzzle ha sido resuelto, todas las casillas
tienen un dígito, y entre todos satisfacen las condiciones señaladas.

En un programa, un sudoku resuelto puede ser guardado en un arreglo de
\(9 × 9\):

\begin{lstlisting}
from numpy import array
sr = array([[4, 2, 6, 5, 7, 1, 3, 9, 8],
            [8, 5, 7, 2, 9, 3, 1, 4, 6],
            [1, 3, 9, 4, 6, 8, 2, 7, 5],
            [9, 7, 1, 3, 8, 5, 6, 2, 4],
            [5, 4, 3, 7, 2, 6, 8, 1, 9],
            [6, 8, 2, 1, 4, 9, 7, 5, 3],
            [7, 9, 4, 6, 3, 2, 5, 8, 1],
            [2, 6, 5, 8, 1, 4, 9, 3, 7],
            [3, 1, 8, 9, 5, 7, 4, 6, 2]])
\end{lstlisting}

Escriba la función \lstinline!solucion_es_correcta(sudoku)! que reciba
como parámetro un arreglo de \(9 × 9\) representando un sudoku resuelto, y
que indique si la solución es correcta (es decir, si no hay elementos
repetidos en filas, columnas y regiones):

\begin{lstlisting}
>>> solucion_es_correcta(s)
True
>>> s[0, 0] = 9
>>> solucion_es_correcta(s)
False
\end{lstlisting}


  \caption{%
    A la izquierda: sudoku sin resolver.
    A la derecha: sudoku resuelto.
  }
  \label{fig:sudoku}
\end{figure}

El \emph{sudoku} es un puzzle que consiste en llenar
una grilla de \(9 × 9\)
(como se ve en la figura~\ref{fig:sudoku})
con los dígitos del 1 al 9,
de modo que no haya ningún valor repetido en
cada fila, en cada columna y en cada uno de las regiones de \(3 × 3\)
marcadas por las líneas más gruesas.

El sudoku sin resolver tiene algunos de los dígitos puestos de antemano
en la grilla. Cuando el puzzle ha sido resuelto, todas las casillas
tienen un dígito, y entre todos satisfacen las condiciones señaladas.

En un programa, un sudoku resuelto puede ser guardado en un arreglo de
\(9 × 9\):

\begin{lstlisting}
from numpy import array
sr = array([[4, 2, 6, 5, 7, 1, 3, 9, 8],
            [8, 5, 7, 2, 9, 3, 1, 4, 6],
            [1, 3, 9, 4, 6, 8, 2, 7, 5],
            [9, 7, 1, 3, 8, 5, 6, 2, 4],
            [5, 4, 3, 7, 2, 6, 8, 1, 9],
            [6, 8, 2, 1, 4, 9, 7, 5, 3],
            [7, 9, 4, 6, 3, 2, 5, 8, 1],
            [2, 6, 5, 8, 1, 4, 9, 3, 7],
            [3, 1, 8, 9, 5, 7, 4, 6, 2]])
\end{lstlisting}

Escriba la función \lstinline!solucion_es_correcta(sudoku)! que reciba
como parámetro un arreglo de \(9 × 9\) representando un sudoku resuelto, y
que indique si la solución es correcta (es decir, si no hay elementos
repetidos en filas, columnas y regiones):

\begin{lstlisting}
>>> solucion_es_correcta(s)
True
>>> s[0, 0] = 9
>>> solucion_es_correcta(s)
False
\end{lstlisting}


  \caption{%
    A la izquierda: sudoku sin resolver.
    A la derecha: sudoku resuelto.
  }
  \label{fig:sudoku}
\end{figure}

El \emph{sudoku} es un puzzle que consiste en llenar
una grilla de \(9 × 9\)
(como se ve en la figura~\ref{fig:sudoku})
con los dígitos del 1 al 9,
de modo que no haya ningún valor repetido en
cada fila, en cada columna y en cada uno de las regiones de \(3 × 3\)
marcadas por las líneas más gruesas.

El sudoku sin resolver tiene algunos de los dígitos puestos de antemano
en la grilla. Cuando el puzzle ha sido resuelto, todas las casillas
tienen un dígito, y entre todos satisfacen las condiciones señaladas.

En un programa, un sudoku resuelto puede ser guardado en un arreglo de
\(9 × 9\):

\begin{lstlisting}
from numpy import array
sr = array([[4, 2, 6, 5, 7, 1, 3, 9, 8],
            [8, 5, 7, 2, 9, 3, 1, 4, 6],
            [1, 3, 9, 4, 6, 8, 2, 7, 5],
            [9, 7, 1, 3, 8, 5, 6, 2, 4],
            [5, 4, 3, 7, 2, 6, 8, 1, 9],
            [6, 8, 2, 1, 4, 9, 7, 5, 3],
            [7, 9, 4, 6, 3, 2, 5, 8, 1],
            [2, 6, 5, 8, 1, 4, 9, 3, 7],
            [3, 1, 8, 9, 5, 7, 4, 6, 2]])
\end{lstlisting}

Escriba la función \lstinline!solucion_es_correcta(sudoku)! que reciba
como parámetro un arreglo de \(9 × 9\) representando un sudoku resuelto, y
que indique si la solución es correcta (es decir, si no hay elementos
repetidos en filas, columnas y regiones):

\begin{lstlisting}
>>> solucion_es_correcta(s)
True
>>> s[0, 0] = 9
>>> solucion_es_correcta(s)
False
\end{lstlisting}

