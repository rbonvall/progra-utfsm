\section{Supermercado}

Un supermercado utiliza tablas de datos para llevar la información de su
inventario.  En un programa, cada tabla de datos es una lista de tuplas.

La lista \lstinline!productos! tiene el código, el nombre, el precio y
la cantidad de unidades del producto en bodega:

\begin{lstlisting}
productos = [
    (41419, 'Fideos',        450, 210),
    (70717, 'Cuaderno',      900, 119),
    (78714, 'Jabon',         730, 708),
    (30877, 'Desodorante',  2190,  79),
    (47470, 'Yogur',          99, 832),
    (50809, 'Palta',         500,  55),
    (75466, 'Galletas',      235,   0),
    (33692, 'Bebida',        700,  20),
    (89148, 'Arroz',         900, 121),
    (66194, 'Lapiz',         120, 900),
    (15982, 'Vuvuzela',    12990,  40),
    (41235, 'Chocolate',    3099,  48),
]
\end{lstlisting}

La lista \lstinline!clientes! tiene el rut y el nombre de los clientes
del supermercado:
\begin{lstlisting}
clientes = [
    ('11652624-7', 'Perico Los Palotes'),
    ( '8830268-0', 'Leonardo Farkas'),
    ( '7547896-8', 'Fulanita de Tal'),
]
\end{lstlisting}

La lista \lstinline!ventas! contiene las ventas realizadas,
representadas por el número de boleta, la fecha de la venta y el rut del
cliente:
\begin{lstlisting}
ventas = [
    (1, (2010,  9, 12),  '8830268-0'),
    (2, (2010,  9, 19), '11652624-7'),
    (3, (2010,  9, 30),  '7547896-8'),
    (4, (2010, 10,  1),  '8830268-0'),
    (5, (2010, 10, 13),  '7547896-8'),
    (6, (2010, 11, 11), '11652624-7'),
]
\end{lstlisting}

El detalle de cada venta se encuentra en la lista \lstinline!itemes!.
Cada ítem tiene asociado un número de boleta, un código de producto y
una cantidad:
\begin{lstlisting}
itemes = [
    (1, 89148,  3), (2, 50809,  4), (2, 33692,  2),
    (2, 47470,  6), (3, 30877,  1), (4, 89148,  1),
    (4, 75466,  2), (5, 89148,  2), (5, 47470, 10),
    (6, 41419,  2),
]
\end{lstlisting}

Por ejemplo, en la venta con boleta número 2, fueron vendidas 4 paltas,
2 bebidas y 6 yogures.

Escriba las siguienes funciones:
\begin{lstlisting}
>>> producto_mas_caro(productos)
'Vuvuzela'
>>> valor_total_bodega(productos)
1900570
>>> ingreso_total_por_ventas(itemes, productos)
13944
>>> producto_con_mas_ingresos(itemes, productos)
'Arroz'
>>> cliente_que_mas_pago(itemes, productos, clientes)
'Fulanita de Tal'
>>> total_ventas_del_mes(2010, 10, itemes, productos)
4160
>>> fecha_ultima_venta_producto(47470, itemes, ventas)
(2010, 10, 13)
\end{lstlisting}

