\section{Fookbace}

\emph{Este problema apareció en el certamen 2 del primer semestre de
2011.}

La red social Fookbace almacena la información de sus usuarios en un
diccionario. Las llaves son un código numérico entero asignado a cada
usuario, y los valores son tuplas con el nombre, la ciudad y la fecha de
nacimiento del usuario. La fecha de nacimiento es una tupla
\lstinline!(año, mes, día)!:

\begin{lstlisting}
usuarios = {
    522514: ('Jean Dupont',        'Marseille',  (1989, 11, 21)),
    587125: ('Perico Los Palotes', 'Valparaiso', (1990,  4, 12)),
    189471: ('Jan Kowalski',       'Krakow',     (1994,  4, 22)),
    914210: ('Antonio Nobel',      'Valparaiso', (1983,  7,  1)),
    # ...
}
\end{lstlisting}

\begin{enumerate}
\item
  Escriba la función \lstinline!misma_ciudad(u1, u2)!, que indique si
  los usuarios con códigos \lstinline!u1! y \lstinline!u2! viven en la
  misma ciudad:

\begin{lstlisting}
>>> misma_ciudad(914210, 587125)
True
>>> misma_ciudad(522514, 189471)
False
\end{lstlisting}
\item
  Escriba la función \lstinline!diferencia_edad(u1, u2)!, que retorne la
  diferencia de edad entre los usuarios cuyos códigos son \lstinline!u1!
  y \lstinline!u2!. (Utilice sólo el año de nacimiento de los usuarios
  para calcular la diferencia, no el mes ni el día).

\begin{lstlisting}
>>> diferencia_edad(914210, 587125)
7
\end{lstlisting}
\item
  Para guardar la información sobre cuáles de sus usuarios son amigos
  entre sí, Fookbace utiliza el conjunto \lstinline!amistades!, que
  contiene tuplas con los códigos de dos usuarios. Si la tupla
  \lstinline!(a, b)! está dentro del conjunto, significa que los
  usuarios con códigos \lstinline!a! y \lstinline!b! son amigos. En
  todas las tuplas se cumple que \lstinline!a! \textless{}
  \lstinline!b!.

\begin{lstlisting}
amistades = {
    (198471, 289142), (138555, 429900), (349123, 781118), # ...
}
\end{lstlisting}

  \begin{enumerate}
  \item
    Escriba la función \lstinline!obtener_amigos(u)!, que retorne el
    conjunto de los códigos de los amigos de \lstinline!u!.
  \item
    Escriba la función \lstinline!recomendar_amigos(u)!, que retorne el
    conjunto de los códigos de los usuarios que cumplen todas estas
    condiciones a la vez:

    \begin{itemize}
    \item
      son amigos de un amigo de \lstinline!u!,
    \item
      no son amigos de \lstinline!u!,
    \item
      viven en la misma ciudad que \lstinline!u!, y
    \item
      tienen menos de diez años de diferencia con \lstinline!u!.
    \end{itemize}
  \end{enumerate}

  En ambas funciones, el parámetro \lstinline!u! es el código de un
  usuario, y el valor de retorno es un conjunto de códigos de usuarios.
  Recuerde que \lstinline!c.add(x)! agrega el valor \lstinline!x! al
  conjunto \lstinline!c!.
\end{enumerate}
