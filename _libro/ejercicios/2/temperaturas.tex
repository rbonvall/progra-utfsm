\section{Reporte de temperaturas}

\emph{Este problema apareció en el certamen 2 del primer semestre de
2011.}

Las temperaturas mínimas y máximas de algunas ciudades de la región
están guardadas en un diccionario cuyas llaves son las ciudades y cuyos
valores son tuplas \lstinline!(mínima, máxima)!.

Se desea generar un archivo cuyo contenido sea un reporte como el del
ejemplo de más abajo. Los nombres de las ciudades en las que hubo más de
25 grados deben aparecer en mayúsculas. El nombre del archivo debe
incluir la fecha. El orden en que aparecen las ciudades dentro del
archivo no importa.

Escriba la función \lstinline!crear_reporte(fecha, temperaturas)!, cuyos
parámetros son la fecha (una tupla \lstinline!(año, mes, día)!) y el
diccionario de temperaturas, y que genere el archivo de texto con el
formato del ejemplo.

La función \lstinline!crear_reporte! no debe retornar nada. Recuerde que
\lstinline!s.upper()! entrega el string \lstinline!s! en mayúsculas.

\begin{lstlisting}
temp = {
  'Vina del Mar':  ( 9, 26),
  'Valparaiso':    (10, 24),
  'Quilpue' :      ( 7, 30),
  'Olmue':         ( 5, 29),
  'Limache':       ( 9, 23),
  'Villa Alemana': ( 9, 22),
}
crear_reporte((2011, 5, 14), temp)
\end{lstlisting}

Archivo \lstinline!reporte-2011-5-14.txt!:

\begin{lstlisting}
QUILPUE: max 30, min 7
Valparaiso: max 24, min 10
VINA DEL MAR: max 26, min 9
Villa Alemana: max 22, min 9
Limache: max 23, min 9
OLMUE: max 29, min 5
\end{lstlisting}

