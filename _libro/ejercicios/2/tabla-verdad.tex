\section{Tabla de verdad}

Un \emph{predicado lógico} es una función cuyos parámetros son
booleanos y su resultado también es booleano.

Escriba la función \lstinline!tabla_de_verdad(predicado)! que reciba
como parámetro un predicado lógico de tres parámetros e imprima la tabla
de verdad del predicado.:

\begin{lstlisting}
>>> def predicado(p, q, r):
...    return (not p) and (q or r)
...
>>> tabla_verdad(predicado)
p     q     r     predicado
===== ===== ===== =========
True  True  True  False
True  True  False False
True  False True  False
True  False False False
False True  True  True
False True  False True
False False True  True
False False False False
\end{lstlisting}

Note que la función \lstinline!tabla_verdad! no retorna nada, sólo
imprime la tabla.
