\section{Mapear y filtrar}

Escriba la función \lstinline!mapear(f, valores)! cuyos parámetros sean
una función \lstinline!f! y una lista \lstinline!valores!, y que retorne
una nueva lista que tenga los elementos obtenidos al aplicar la función
a los elementos de la lista:

\begin{lstlisting}
>>> def cuadrado(x):
...     return x ** 2
...
>>> mapear(cuadrado, [5, 2, 9])
[25, 4, 81]
\end{lstlisting}

A continuación,
escriba la función \lstinline!filtrar(f, valores)! cuyos pará\-metros sean
una función \lstinline!f! que retorne un valor booleano y una lista
\lstinline!valores!, y que retorne una nueva lista que tenga todos los
elementos de \lstinline!valores! para los que la función \lstinline!f!
retorna \lstinline!True!:

\begin{lstlisting}
>>> def es_larga(palabra):
...     return len(palabra) > 4
...
>>> p = ['arroz', 'leon', 'oso', 'mochila']
>>> filtrar(es_larga, p)
['arroz', 'mochila']
\end{lstlisting}

Las funciones no deben modificar la lista original, sino retornar una
nueva:
\begin{lstlisting}
>>> filtrar(es_larga, p)
['arroz', 'mochila']
>>> p
['arroz', 'leon', 'oso', 'mochila']
\end{lstlisting}

(En Python, estas funciones ya existen, y se llaman \lstinline!map! y
\lstinline!filter!. Haga como que no lo sabe y escriba las funciones por
su cuenta).
