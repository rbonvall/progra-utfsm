\section{Personas}

Para realizar estos ejercicios , usted debe descargar
\href{../../\_static/personas.py}{el módulo con los datos} que vamos a
utilizar.

Para usar el módulo hay que descargarlo en la misma carpeta en la que se
guardará el programa e importar los datos de esta forma:

\begin{lstlisting}
from personas import *
\end{lstlisting}

Este módulo contiene una lista llamada \lstinline!personas! que contiene
tuplas que representan los datos de una persona. Cada tupla tiene tres
valores: el nombre, el apellido y la fecha de nacimiento.

El nombre y el apellido son strings, y la fecha de nacimiento es una
tupla de tres valores: el día, el mes y el año.

Por ejemplo, podemos ver los datos de la primera persona:

\begin{lstlisting}
>>> personas[0]
('Martín', 'Soto', (24, 8, 1990))
\end{lstlisting}

\begin{enumerate}
\item
  Escriba una función que imprima el nombre de todas las personas. Para
  eso, recorra la lista con un \lstinline!for!, obtenga el nombre de la
  persona e imprímalo usando \lstinline!print!. La función no tiene que
  retornar nada:

\begin{lstlisting}
>>> imprimir_nombres(personas)
Martín
Gabriel
Humberto
Sebastián
Víctor
...
Horacio
Ignacio
Nicolás
Pablo
Rolando
Ricardo
\end{lstlisting}
\item
  Escriba una función que imprima la fecha de nacimiento de todas las
  personas:

\begin{lstlisting}
>>> imprimir_fechas(personas)
24 de agosto de 1990
2 de junio de 1974
14 de noviembre de 1973
18 de septiembre de 1973
12 de agosto de 1992
...
18 de agosto de 1981
24 de abril de 1972
17 de mayo de 1977
4 de febrero de 1972
29 de enero de 1976
\end{lstlisting}

  Para hacerlo más fácil, construya un diccionario con los nombres de
  los meses:

\begin{lstlisting}
meses = {
    1: 'enero',
    2: 'febrero',
    # ...
    12: 'diciembre',
}
\end{lstlisting}
\item
  Escriba una función llamada \lstinline!cuantas_personas(personas)! que
  retorne la cantidad de personas en la lista.
\item
  Escriba una función que retorne la lista de las personas que tienen
  cumpleaños el mismo día que usted. Por ejemplo:

\begin{lstlisting}
>>> mi_cumple(personas)
['Jonathan Sepulveda']
\end{lstlisting}
\item
  Escriba una función llamada \lstinline!cumples_repetidos(personas)!
  que pueda determinar las personas en la lista que tienen su cumpleaños
  el mismo día.
\item
  Escriba una función llamada \lstinline!nombre_mas_comun(personas)! que
  entregue el nombre que más se repite.
\end{enumerate}
