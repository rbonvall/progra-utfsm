\section{Traslape de rectángulos}

Un rectángulo que está en el plano \emph{xy} cuyos lados son paralelos a
los ejes cartesianos puede ser representado mediante cuatro datos:

\begin{itemize}
\item
  la coordenada \emph{x} de su vértice inferior izquierdo,
\item
  la coordenada \emph{y} de su vértice inferior izquierdo.
\item
  su ancho \emph{w}, y
\item
  su altura \emph{h}.
\end{itemize}

\includegraphics{../../diagramas/rect.png}

En un programa en Python, esto se traduce en una tupla
\lstinline!(x, y, w, h)! de cuatro elementos:

\begin{lstlisting}
# el rectangulo de la figura
rectangulo = (3, 2, 5, 6)
\end{lstlisting}

\begin{enumerate}
\item
  Escriba la función \lstinline!ingresar_rectangulo()! que pida al
  usuario ingresar los datos de un rectángulo, y retorne la tupla con
  los datos ingresados. La función no tiene parámetros. Al ejecutar la
  función, la sesión debe verse así:

  Con esta entrada, la función retornaría la tupla
  \lstinline!(3, 2, 5, 6)!.
\item
  Escriba la función \lstinline!se_traslapan(r1, r2)! que reciba como
  parámetros dos rectángulos \lstinline!r1! y \lstinline!r2!, y entregue
  como resultado si los rectángulos se traslapan o no.

  Por ejemplo, en el siguiente diagrama, los rectángulos A y B se
  traslapan, mientras que los rectángulos A y C no se traslapan:

  \includegraphics{../../diagramas/rect2.png}

\begin{lstlisting}
>>> a = (1, 8, 8, 5)
>>> b = (7, 6, 3, 6)
>>> c = (4, 2, 9, 3)
>>> se_traslapan(a, b)
True
>>> se_traslapan(b, c)
False
>>> se_traslapan(a, c)
False
\end{lstlisting}
\item
  Escriba un programa que pida al usuario ingresar varios rectángulos, y
  termine cuando se ingrese uno que se traslape con alguno de los
  ingresados anteriormente. La salida debe indicar cuáles son los
  rectángulos que se traslapan.
\item
  (¡Difícil!). Escriba la función
  \lstinline!contar_regiones_continuas(rectangulos)! que reciba como
  parámetro una lista de rectángulos, y retorne la cantidad de regiones
  continuas formadas por rectángulos traslapados.

  Por ejemplo, en el siguiente diagrama hay 15 rectángulos que forman 6
  regiones continuas de rectángulos traslapados:

  \includegraphics{../../diagramas/rect3.png}

  Los rectángulos de la figura son los siguientes:

\begin{lstlisting}
rs = [
    ( 4,  2, 9, 3),
    (14, 10, 5, 1),
    (14, 17, 3, 2),
    (13,  7, 2, 2),
    ( 8, 16, 4, 3),
    (13, 14, 2, 4),
    ( 1,  8, 8, 5),
    ( 1,  1, 6, 4),
    (16, 14, 3, 4),
    (12,  6, 4, 6),
    ( 7,  6, 3, 6),
    ( 5, 15, 4, 3),
    (14, 13, 3, 2),
    (15,  3, 5, 4),
    ( 2, 16, 3, 3),
]
\end{lstlisting}

  Puede usar esta lista para probar su función:

\begin{lstlisting}
>>> contar_regiones_continuas(rs)
6
\end{lstlisting}
\end{enumerate}
