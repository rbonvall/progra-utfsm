\section{Gato}

Desarrolle un programa para jugar al gato determinando \emph{la mejor}
jugada del jugador que juega con las \emph{X}.

Considere que todas las celdas están vacías y evalúe las jugadas
potenciales en ellas utilizando la siguiente estrategia:

\begin{itemize}
\item
  Si la jugada ocupa el tercer cuadro en una fila, columna o diagonal
  que tenga ya dos X, sume 50 al marcador.
\item
  Si la jugada ocupa el tercer cuadro de una fila, columna o diagonal
  con dos O, sume 25 al marcador.
\item
  Si tras la jugada una fila, columna o diagonal contiene dos X y un
  blanco, sume 10.
\item
  Sume 8 si una fila, columna o diagonal queda, después de la jugada,
  con una O, una X o un blanco.
\item
  Sume 4 por cada fila, columna o diagonal que quede con una X y el
  resto blancos.
\end{itemize}

Seleccione la jugada de mayor marcador.

Considerando el siguiente tablero:

\includegraphics{../../diagramas/gato.png}

Las posibles jugadas y sus marcadores son los siguientes:

\ctable[pos = H, center, botcap]{ll}
{% notes
}
{% rows
\FL
\parbox[b]{0.15\columnwidth}{\raggedright
Posición
} & \parbox[b]{0.18\columnwidth}{\raggedright
Marcador
}
\ML
\parbox[t]{0.15\columnwidth}{\raggedright
\begin{quote}
1
\end{quote}
} & \parbox[t]{0.18\columnwidth}{\raggedright
10+8=18
}
\\\noalign{\medskip}
\parbox[t]{0.15\columnwidth}{\raggedright
\begin{quote}
2
\end{quote}
} & \parbox[t]{0.18\columnwidth}{\raggedright
10+8=18
}
\\\noalign{\medskip}
\parbox[t]{0.15\columnwidth}{\raggedright
\begin{quote}
3
\end{quote}
} & \parbox[t]{0.18\columnwidth}{\raggedright
10+10=20
}
\\\noalign{\medskip}
\parbox[t]{0.15\columnwidth}{\raggedright
\begin{quote}
4
\end{quote}
} & \parbox[t]{0.18\columnwidth}{\raggedright
\begin{quote}
8=8
\end{quote}
}
\\\noalign{\medskip}
\parbox[t]{0.15\columnwidth}{\raggedright
\begin{quote}
5
\end{quote}
} & \parbox[t]{0.18\columnwidth}{\raggedright
10+10+8=28
}
\LL
}

En este caso se elige la jugada 5.
