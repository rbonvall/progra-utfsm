\section{Distancias}

La siguiente tabla indica las distancias entre algunas ciudades
chilenas:

\begin{quote}
\ctable[pos = H, center, botcap]{llllll}
{% notes
}
{% rows
\FL
\textbf{Distancia} & Arica & Santiago & Valparaíso & San
Fernando & Temuco
\\\noalign{\medskip}
Arica & 0 & 2050 & 2015 & 2190 & 2725
\\\noalign{\medskip}
Santiago & 2050 & 0 & 119 & 140 & 675
\\\noalign{\medskip}
Valparaíso & 2015 & 119 & 0 & 255 & 790
\\\noalign{\medskip}
San Fernando & 2190 & 140 & 255 & 0 & 535
\\\noalign{\medskip}
Temuco & 2725 & 675 & 790 & 535 & 0
\LL
}
\end{quote}

En un programa, esta información puede representarse mediante un arreglo
de ciudades y un arreglo bidimensional de distancias.

\begin{itemize}
\item
  Escriba la función que pregunte al usuario la cantidad de ciudades y
  los nombres de las ciudades:

\begin{lstlisting}
¿Cuántas ciudades?
\end{lstlisting}

  \begin{quote}
  \begin{description}
  \item[5]
  Ingrese los nombres:
  \end{description}

  Arica Santiago Valparaíso San Fernando Temuco
  \end{quote}
\item
  Escriba la función que pregunte al usuario las distancias entre cada
  par de ciudades:

\begin{lstlisting}
¿Distancia Arica-Santiago?
\end{lstlisting}

  \begin{quote}
  \begin{description}
  \item[2050]
  ¿Distancia Arica-Valparaíso?
  \end{description}

  2015 \ldots{} ¿Distancia Temuco-San Fernando? 535
  \end{quote}

  Tener la precaución de preguntar sólo una vez cada par de ciudades: si
  se pregunta Arica-Santiago, no debe preguntarse Santiago-Arica.
  Tampoco debe preguntarse la distancia desde una ciudad a sí misma.
\item
  Escriba la función que pida al usuario que ingrese una lista de
  ciudades, que representan un itinerario por realizar:

\begin{lstlisting}
¿Cuántas ciudades tiene el itinerario?
\end{lstlisting}

  \begin{quote}
  \begin{description}
  \item[4]
  Ingrese las ciudades del itinerario:
  \end{description}

  Temuco Santiago San Fernando Arica
  \end{quote}
\item
  Escriba la función \lstinline!kms! que entregue como resultado los
  kilómetros que hay que recorrer para visitar las ciudades en orden.
  Por ejemplo, para el itinerario de arriba, debe entregar como
  resultado `3005`.
\end{itemize}
