\section{Números de Fibonacci}

Los
\href{http://es.wikipedia.org/wiki/N\%C3\%BAmeros\_de\_Fibonacci}{números
de Fibonacci} `F\_k` son una sucesión de números naturales definidos de
la siguiente manera:

\[F_0 &= 0, \\
F_1 &= 1, \\
F_k &= F_{k - 1} + F_{k - 2}, \qquad\text{cuando } k\ge 2.\]

En palabras simples, la sucesión de Fibonacci comienza con 0 y 1, y los
siguientes términos siempre son la suma de los dos anteriores.

En la siguiente tabla, podemos ver los números de Fibonacci desde el
0-ésimo hasta el duodécimo.

\ctable[pos = H, center, botcap]{lllllllllllllll}
{% notes
}
{% rows
\FL
$n$ & 0 & 1 & 2 & 3 & 4 & 5 & 6 & 7 & 8 & 9 & 10 & 11 & 12 & \ldots{}
\\\noalign{\medskip}
$F_n$ & 0 & 1 & 1 & 2 & 3 & 5 & 8 & 13 & 21 & 34 & 55 & 89 & 144 & \ldots{}
\LL
}

\begin{enumerate}
\item
  Escriba un programa que reciba como entrada un número entero \emph{n},
  y entregue como salida el \emph{n}-ésimo número de Fibonacci:
\item
  Escriba un programa que reciba como entrada un número entero e indique
  si es o no un número de Fibonacci:
\item
  Escriba un programa que muestre los \emph{m} primeros números de
  Fibonacci, donde \emph{m} es un número ingresado por el usuario:
\end{enumerate}
