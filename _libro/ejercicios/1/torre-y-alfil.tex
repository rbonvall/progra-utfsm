\section{Torre y alfil}

\emph{Este problema apareció en el certamen 1 del primer semestre de
2011.}

Un tablero de ajedrez es una grilla de ocho filas y ocho columnas,
numeradas de 1 a 8. Dos de las piezas del juego de ajedrez son el alfil
y la torre. El alfil se desplaza en diagonal, mientras que la torre se
desplaza horizontal o verticalmente. Una pieza puede ser capturada por
otra si está en una casilla a la cual la otra puede desplazarse:

\includegraphics{../../diagramas/torre-alfil.png}

Escriba un programa que reciba como entrada las posiciones en el tablero
de un alfil y de una torre, e indique cuál pieza captura a la otra:

Suponga que los datos ingresados son válidos. Su programa debe funcionar
para tableros de cualquier tamaño.
