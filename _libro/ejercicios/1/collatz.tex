\section{Secuencia de Collatz}

La secuencia de Collatz de un número entero se construye de la siguiente
forma:

\begin{itemize}
\item
  si el número es par, se lo divide por dos;
\item
  si es impar, se le multiplica tres y se le suma uno;
\item
  la sucesión termina al llegar a uno.
\end{itemize}

La \href{http://es.wikipedia.org/wiki/Conjetura\_de\_Collatz}{conjetura
de Collatz} afirma que, al partir desde cualquier número, la secuencia
siempre llegará a 1. A pesar de ser una afirmación a simple vista muy
simple, no se ha podido demostrar si es cierta o no.

Usando computadores, se ha verificado que la sucesión efectivamente
llega a 1 partiendo desde cualquier número natural menor que
\(2^{58}\).

Desarrolle un programa que entregue la secuencia de Collatz de un
número entero:

\begin{lstlisting}[language=testcase]
n: `18`
18 9 28 14 7 22 11 34 17 52 26 13 40 20 10 5 16 8 4 2 1
\end{lstlisting}

\begin{lstlisting}[language=testcase]
n: `19`
19 58 29 88 44 22 11 34 17 52 26 13 40 20 10 5 16 8 4 2 1
\end{lstlisting}

\begin{lstlisting}[language=testcase]
n: `20`
20 10 5 16 8 4 2 1
\end{lstlisting}

A continuación, desarrolle un programa que grafique los largos de las secuencias de
Collatz de los números enteros positivos menores que el ingresado por
el usuario:

\begin{lstlisting}[language=testcase]
n: `10`
1 *
2 **
3 ********
4 ***
5 ******
6 *********
7 *****************
8 ****
9 ********************
10 *******
\end{lstlisting}

