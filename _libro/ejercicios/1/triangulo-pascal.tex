\section{Triángulo de Pascal}

\begin{figure}
  \centering
  \begin{tikzpicture}[xscale=1.2, yscale=0.8]

  % Thanks to Paul Gaborit for this code.
  % http://www.texample.net/tikz/examples/pascals-triangle-and-sierpinski-triangle/

  % Pascal's triangle
  % row #0 => value is 1
  \node (p-0-0) at (0,0) {1};
  \foreach \pascalrow in {1,...,6} {
     % col #0 => value is 1
    \node (p-\pascalrow-0) at (-\pascalrow/2,-\pascalrow) {1};
    \pgfmathsetmacro{\pascalvalue}{1};
    \foreach \pascalcol in {1,...,\pascalrow} {
      % iterative formula : val = precval * (row-col+1)/col
      % (+ 0.5 to bypass rounding errors)
      \pgfmathtruncatemacro{\pascalvalue}{\pascalvalue*((\pascalrow-\pascalcol+1)/\pascalcol)+0.5};
      \global\let\pascalvalue=\pascalvalue
      % position of each value
      \coordinate (pascalpos) at (-\pascalrow/2+\pascalcol,-\pascalrow);
      \node (p-\pascalrow-\pascalcol) at (pascalpos) {\pascalvalue};
      % for arrows and plus sign
      \ifnum \pascalcol<\pascalrow
        \node[Red, above=0mm of p-\pascalrow-\pascalcol]{+};
        \pgfmathtruncatemacro{\pascalprow}{\pascalrow-1}
        \pgfmathtruncatemacro{\pascalpcol}{\pascalcol-1}
        \draw[Red, -latex'] (p-\pascalprow-\pascalpcol) -- (p-\pascalrow-\pascalcol);
        \draw[Red, -latex'] (p-\pascalprow-\pascalcol) -- (p-\pascalrow-\pascalcol);
      \fi
    }
  }
\end{tikzpicture}


  \caption{Triángulo de Pascal}
  \label{fig:triangulo-pascal}
\end{figure}

Desarrolle un programa que dibuje un
triángulo de Pascal, o sea, una disposición de números enteros tales que cada uno
sea la suma de los dos que están por encima de él,
tal como se ve en la figura~\ref{fig:triangulo-pascal}.

Un programa que genere las primeras cinco líneas
podría verse así:
\begin{lstlisting}[language=testcase]
1
1  1
1  2  1
1  3  3  1
1  4  6  4  1
1  5 10 10  5  1
\end{lstlisting}
o así:
\begin{lstlisting}[language=testcase]
          1
        1   1
      1   2   1
    1   3   3   1
  1   4   6   4   1
1   5  10  10   5   1
\end{lstlisting}
Usted, empero, debe generar las primeras veinte líneas.
Considere que en la línea 20 aparecen números de cinco dígitos.

