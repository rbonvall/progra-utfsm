\section{Más corta y más larga}

Desarrolle un programa que tenga la siguiente entrada:

\begin{itemize}
\item
  primero, el usuario ingresa un número entero \(n\), que indica
  cuántas palabras ingresará a continuación;
\item
  después el usuario ingresa \(n\) palabras.
\end{itemize}

La salida del programa debe mostrar la palabra más larga y la más corta
que fueron ingresadas por el usuario.

Recuerde que la función \lstinline!len! entrega el largo de un string:

\begin{lstlisting}
>>> len('amarillo')
8
\end{lstlisting}

La ejecución del programa debe verse así:
\begin{lstlisting}[language=testcase]
Cantidad de palabras: `5`
Palabra 1: `negro`
Palabra 2: `amarillo`
Palabra 3: `naranjo`
Palabra 4: `azul`
Palabra 5: `blanco`
La palabra mas larga es amarillo
La palabra mas corta es azul
\end{lstlisting}


