\chapter{Sentencias de control}

Un programa es una sucesión de \textbf{sentencias} que son ejecutadas
secuencialmente.
Por ejemplo, el siguiente programa tiene cuatro sentencias:
\begin{lstlisting}
n = int(raw_input('Ingrese n: '))
m = int(raw_input('Ingrese m: '))
suma = n + m
print 'La suma de n y m es:', suma
\end{lstlisting}

Las primeras tres son asignaciones, y la última es una llamada a
función. Al ejecutar el programa, cada una de estas sentencias es
ejecutada, una después de la otra, una sola vez.

Además de las sentencias simples, que son ejecutadas en secuencia,
existen las \textbf{sentencias de control} que permiten modificar el
flujo del programa introduciendo ciclos y condicionales.

Un \textbf{condicional} es un conjunto de sentencias que pueden o no
ejecutarse, dependiendo del resultado de una condición.

Un \textbf{ciclo} es un conjunto de sentencias que son ejecutadas varias
veces, hasta que una condición de término es satisfecha.

Tanto los condicionales como los ciclos contienen a otras sentencias.
Para indicar esta relación se utiliza la \textbf{indentación}: las
sentencias contenidas no se escriben en la misma columna que la
sentencia de control, sino un poco más a la derecha:

\begin{lstlisting}
n = int(raw_input())
m = int(raw_input())
if m < n:
    t = m
    m = n
    n = t
print m, n
\end{lstlisting}

En este ejemplo, las tres asignaciones están contenidas dentro de la
sentencia de control \lstinline!if!. El \lstinline!print m, n! no está
indentado, por lo que no es parte de la sentencia \lstinline!if!.

Este programa tiene cuatro sentencias, de las cuales la tercera es una
sentencia de control, que contiene a otras tres sentencias.

Para indentar, utilizaremos siempre cuatro espacios.

\section{Condicional if}

La sentencia \textbf{if} («si») ejecuta las instrucciones sólo si se
cumple una condición. Si la condición es falsa, no se hace nada:

\begin{center}
  \documentclass{minimal}
\usepackage[pdftex,active,tightpage]{preview}
\usepackage[utf8]{inputenc}
\usepackage[spanish]{babel}
\usepackage{mathpazo}
\usepackage{tikz}
\usetikzlibrary{calc,shapes,arrows}
\PreviewEnvironment{tikzpicture}

\begin{document}
\tikzstyle{decision} = [
  diamond,
  very thick,
  draw=red!50!black!50,
  %fill=red!20, 
  aspect=2,
  %text badly centered,
  top color=white,
  bottom color=red!50!black!20,
]
\tikzstyle{stmt} = [
  rectangle,
  very thick,
  draw=blue!50!black!50,
  %fill=blue!20, 
  text centered,
  minimum height=5ex,
  minimum width=5em,
  top color=white,
  bottom color=blue!50!black!20,
]
\tikzstyle{node} = [
  circle,
  very thick,
  draw=orange!50!black!50,
  fill=orange!20,
  minimum size=6ex,
]
\tikzstyle{io} = [
  very thick,
  draw=green!50!black!50,
  trapezium,
  trapezium left angle=80,
  trapezium right angle=-80,
  %fill=green!20!black!10,
  %rounded corners,
  %minimum height=5ex,
  top color=white,
  bottom color=green!50!black!20,
  text centered,
  minimum height=5ex,
  minimum width=5em,
]
\tikzstyle{conn} = [very thick, draw=black!50, -latex']


\begin{tikzpicture}[node distance=12ex, auto]
    \node (start) { };
    \node [decision, below of=start] (dcs) { condición };
    \node [stmt, below of=dcs] (true) { caso verdadero };
    \node [below of=true] (end) { };

    \path [conn] (start) -- (dcs);
    \path [conn] (dcs) --
                 node[very near start] { sí }
                 (true);
    \path [conn] (true) -- (end);
    \path [conn] (dcs.east) --
                 node[very near start] { no }
                 +(1, 0) |-
                 ($ (true.south)!.5!(end.north) $);
\end{tikzpicture}
\end{document}

\end{center}


La sintaxis es la siguiente:

\begin{lstlisting}
if condicion:
    sentencias
\end{lstlisting}

Por ejemplo, el siguente programa felicita a alguien que aprobó el ramo:

\begin{lstlisting}
nota = int(raw_input('Ingrese su nota: '))
if nota >= 55:
    print 'Felicitaciones'
\end{lstlisting}

Ejecute este programa, probando varias veces con valores diferentes.

\section{Condicional if-else}

La sentencia \textbf{if-else} («si-o-si-no») decide qué instrucciones
ejecutar dependiendo si una condición es verdadera o falsa:

\begin{center}
  \begin{tikzpicture}[node distance=12ex, auto]
    \node (start) { };
    \node [decision, below of=start] (dcs) { condición };
    \node [stmt, below of=dcs] (true) { caso verdadero };
    \node [stmt, right of=true, node distance=7em] (false) { caso falso };
    \node [below of=true] (end) { };

    \path [conn] (start) -- (dcs);
    \path [conn] (dcs) --
                 node[very near start] { sí }
                 (true);
    \path [conn] (true) -- (end);
    \path [conn] (dcs.east) -|
                 node[very near start] { no }
                 (false.north);
    \path [conn] (false.south) |-
                 ($ (true.south)!.5!(end.north) $);
\end{tikzpicture}

\end{center}


La sintaxis es la siguiente:

\begin{lstlisting}
if condicion:
    que hacer cuando la condicion es verdadera
else:
    que hacer cuando la condicion es falsa
\end{lstlisting}

Por ejemplo, el siguiente programa indica a alguien si es mayor de edad:

\begin{lstlisting}
edad = int(raw_input('Cual es su edad? '))
if edad < 18:
    print 'Usted es menor de edad'
else:
    print 'Usted es adulto'
\end{lstlisting}

El siguiente programa realiza acciones distintas dependiendo de si el
número de entrada es par o impar:

\begin{lstlisting}
n = int(raw_input('Ingrese un numero: '))
if n % 2 == 0:
    print 'El numero es par'
    print 'La mitad del numero es', n / 2
else:
    print 'El numero es impar'
    print 'El sucesor del numero es', n + 1
print 'Listo'
\end{lstlisting}

La última sentencia no está indentada, por lo que no es parte del
condicional, y será ejecutada siempre.

\section{Condicional if-elif-else}

La sentencia \textbf{if-elif-else} depende de dos o más condiciones, que
son eva\-lua\-das en orden. La primera que es verdadera determina qué
instrucciones serán ejecutadas:

\begin{center}
  \documentclass{minimal}
\usepackage[pdftex,active,tightpage]{preview}
\usepackage[utf8]{inputenc}
\usepackage[spanish]{babel}
\usepackage{mathpazo}
\usepackage{tikz}
\usetikzlibrary{calc,shapes,arrows}
\PreviewEnvironment{tikzpicture}

\begin{document}
\tikzstyle{decision} = [
  diamond,
  very thick,
  draw=red!50!black!50,
  %fill=red!20, 
  aspect=2,
  %text badly centered,
  top color=white,
  bottom color=red!50!black!20,
]
\tikzstyle{stmt} = [
  rectangle,
  very thick,
  draw=blue!50!black!50,
  %fill=blue!20, 
  text centered,
  minimum height=5ex,
  minimum width=5em,
  top color=white,
  bottom color=blue!50!black!20,
]
\tikzstyle{node} = [
  circle,
  very thick,
  draw=orange!50!black!50,
  fill=orange!20,
  minimum size=6ex,
]
\tikzstyle{io} = [
  very thick,
  draw=green!50!black!50,
  trapezium,
  trapezium left angle=80,
  trapezium right angle=-80,
  %fill=green!20!black!10,
  %rounded corners,
  %minimum height=5ex,
  top color=white,
  bottom color=green!50!black!20,
  text centered,
  minimum height=5ex,
  minimum width=5em,
]
\tikzstyle{conn} = [very thick, draw=black!50, -latex']


\begin{tikzpicture}[node distance=12ex, auto]
    \node (start) { };
    \node [decision, below of=start] (dcs1) { condición 1 };
    \node [decision, below of=dcs1] (dcs2)  { condición 2 };
    \node [decision, below of=dcs2] (dcs3)  { condición 3 };
    \node [stmt, right of=dcs1, node distance=8em] (true1) { caso 1 };
    \node [stmt, right of=dcs2, node distance=8em] (true2) { caso 2 };
    \node [stmt, right of=dcs3, node distance=8em] (true3) { caso 3 };
    \node [stmt, below of=dcs3] (false) { caso falso };
    \node [below of=false] (end) { };

    \path [conn] (start) -- (dcs1);
    \path [conn] (dcs1) -- node [near start] { no } (dcs2);
    \path [conn] (dcs2) -- node [near start] { no } (dcs3);
    \path [conn] (dcs3) -- node [near start] { no } (false);
    \path [conn] (dcs1) -- node [near start] { sí } (true1);
    \path [conn] (dcs2) -- node [near start] { sí } (true2);
    \path [conn] (dcs3) -- node [near start] { sí } (true3);
    \path [conn] (true1) -- +(1.5, 0) |- ($ (false.south)!.5!(end)$);
    \path [conn] (true2) -- +(1.5, 0);
    \path [conn] (true3) -- +(1.5, 0);
    \path [conn] (false) -- (end);
\end{tikzpicture}
\end{document}

\end{center}

La sintaxis es la siguiente:

\begin{lstlisting}
if condicion1:
    que hacer si condicion1 es verdadera
elif condicion2:
    que hacer si condicion2 es verdadera
...
else:
    que hacer cuando ninguna de las
    condiciones anteriores es verdadera
\end{lstlisting}

El último \lstinline!else! es opcional.

\begin{table}
  \centering
  \begin{tabular}{lr}
    \toprule
      Sueldo & Tasa de impuesto \\
    \midrule
      menos de \(1000\)                 &  0\% \\
      \(1000 \le \text{sueldo} < 2000\) &  5\% \\
      \(2000 \le \text{sueldo} < 4000\) & 10\% \\
      \(4000\) o más                    & 12\% \\
    \bottomrule
  \end{tabular}
  \caption{Ejemplo: tasa de impuesto en función del sueldo recibido.}
  \label{tbl:tasa-impuesto}
\end{table}

Por ejemplo, la tasa de impuesto a pagar por una persona según su sueldo
puede estar dada por la tabla~\ref{tbl:tasa-impuesto}.
Entonces, el programa que calcula el impuesto a pagar es el siguiente:

\begin{lstlisting}
sueldo = int(raw_input('Ingrese su sueldo: '))
if sueldo < 1000:
    tasa = 0.00
elif sueldo < 2000:
    tasa = 0.05
elif sueldo < 4000:
    tasa = 0.10
else:
    tasa = 0.12
print 'Usted debe pagar', tasa * sueldo, 'de impuesto'
\end{lstlisting}

Siempre sólo una de las alternativas será ejecutada. Tan pronto alguna de las
condiciones es verdadera, el resto de ellas no siguen siendo evaluadas.

Otra manera de escribir el mismo programa usando sólo sentencias
\lstinline!if! es la siguiente:

\begin{lstlisting}
sueldo = int(raw_input('Ingrese su sueldo: '))
if sueldo < 1000:
    tasa = 0.00
if 1000 <= sueldo < 2000:
    tasa = 0.05
if 2000 <= sueldo < 4000:
    tasa = 0.10
if 4000 < sueldo:
    tasa = 0.12
print 'Usted debe pagar', tasa * sueldo, 'de impuesto'
\end{lstlisting}

Esta manera es menos clara, porque no es evidente a primera vista que
sólo una de las condiciones será verdadera.

\section{Ciclo while}

El ciclo \textbf{while} («mientras») ejecuta una secuencia de
instrucciones mientras una condición sea verdadera:

\begin{center}
  \documentclass{minimal}
\usepackage[pdftex,active,tightpage]{preview}
\usepackage[utf8]{inputenc}
\usepackage[spanish]{babel}
\usepackage{mathpazo}
\usepackage{tikz}
\usetikzlibrary{calc,shapes,arrows}
\PreviewEnvironment{tikzpicture}

\begin{document}
\tikzstyle{decision} = [
  diamond,
  very thick,
  draw=red!50!black!50,
  %fill=red!20, 
  aspect=2,
  %text badly centered,
  top color=white,
  bottom color=red!50!black!20,
]
\tikzstyle{stmt} = [
  rectangle,
  very thick,
  draw=blue!50!black!50,
  %fill=blue!20, 
  text centered,
  minimum height=5ex,
  minimum width=5em,
  top color=white,
  bottom color=blue!50!black!20,
]
\tikzstyle{node} = [
  circle,
  very thick,
  draw=orange!50!black!50,
  fill=orange!20,
  minimum size=6ex,
]
\tikzstyle{io} = [
  very thick,
  draw=green!50!black!50,
  trapezium,
  trapezium left angle=80,
  trapezium right angle=-80,
  %fill=green!20!black!10,
  %rounded corners,
  %minimum height=5ex,
  top color=white,
  bottom color=green!50!black!20,
  text centered,
  minimum height=5ex,
  minimum width=5em,
]
\tikzstyle{conn} = [very thick, draw=black!50, -latex']


\begin{tikzpicture}[node distance=12ex, auto]
    \node (start) { };
    \node [decision, below of=start] (cond) { condición };
    \node [stmt, right of=cond, node distance=8em] (iter) { iteración };
    \node [below of=cond] (end) { };

    \path [conn] (start) -- (cond);
    \path [conn] (cond) --
                 node[very near start] { sí }
                 (iter);
    \path [conn] (iter.north) |-
                 ($ (start)!.5!(cond.north) $);
    \path [conn] (cond) --
                 node[very near start] { no }
                 (end);
\end{tikzpicture}
\end{document}

\end{center}

Cada una de las veces que el cuerpo del ciclo es ejecutado se llama
\textbf{iteración}.

La condición es evaluada antes de cada iteración. Si la condición es
inicialmente falsa, el ciclo no se ejecutará ninguna vez.

La sintaxis es la siguiente:

\begin{lstlisting}
while condicion:
    sentencias
\end{lstlisting}

Por ejemplo, el siguiente programa multiplica dos números enteros sin
usar el operador \lstinline!*!:

\begin{lstlisting}
m = int(raw_input())
n = int(raw_input())
p = 0
while m > 0:
    m = m - 1
    p = p + n
print 'El producto de m y n es', p
\end{lstlisting}

\begin{table}
  \centering
  \begin{tabular}{*{3}{r}}
    \toprule
      \lstinline!p! & \lstinline!m! & \lstinline!n! \\
    \midrule
         &  4 &   \\
         &    & 7 \\
       0 &    &   \\
         &  3 &   \\
       7 &    &   \\
         &  2 &   \\
      14 &    &   \\
         &  1 &   \\
      21 &    &   \\
         &  0 &   \\
      28 &    &   \\
    \bottomrule
  \end{tabular}
  \caption{%
    Ruteo del programa de ejemplo.
    La tabla muestra cómo cambian los valores de cada variable
    durante la ejecución del programa.
  }
  \label{tbl:ruteo-while}
\end{table}

Para ver cómo funciona este programa, hagamos un ruteo con la entrada
\lstinline!m! = 4 y \lstinline!n! = 7.
La tabla~\ref{tbl:ruteo-while} muestra cómo cambian
los valores de todas las variables
a medida que las sentencias van siendo ejecutadas.

En cada iteración, el valor de \lstinline!m! decrece en 1. Cuando llega
a 0, la condición del \lstinline!while! deja de ser verdadera por lo que
el ciclo termina. De este modo, se consigue que el resultado sea sumar
\lstinline!m! veces el valor de \lstinline!n!.

Note que el ciclo no termina apenas el valor de \lstinline!m! pasa a ser
cero. La condición es evaluada una vez que la iteración completa ha
terminado.

En general, el ciclo \lstinline!while! se utiliza cuando no es posible
saber de antemano cuántas veces será ejecutado el ciclo, pero sí qué es
lo que tiene que ocurrir para que se termine.

\section{Ciclo for con rango}

El ciclo \textbf{for con rango} ejecuta una secuencia de sentencias una
cantidad fija de veces.

Para llevar la cuenta, utiliza una \textbf{variable de control} que toma
valores distintos en cada iteración.

Una de las sintaxis para usar un \lstinline!for! con rango es la
siguiente:

\begin{lstlisting}
for variable in range(fin):
    que hacer para cada valor de la variable de control
\end{lstlisting}

En la primera iteración, la variable de control toma el valor 0. Al
final de cada iteración, el valor de la variable aumenta
automáticamente. El ciclo termina justo antes que la variable tome el
valor \lstinline!fin!.

Por ejemplo, el siguiente programa muestra los cubos de los números del
0 al 20:

\begin{lstlisting}
for i in range(21):
    print i, i ** 3
\end{lstlisting}

Un \textbf{rango} es una sucesión de números enteros equiespaciados.
Incluyendo la presentada más arriba, hay tres maneras de definir un
rango:

\begin{lstlisting}
range(final)
range(inicial, final)
range(inicial, final, incremento)
\end{lstlisting}

El valor inicial siempre es parte del rango. El valor final nunca es
incluido en el rango. El incremento indica la diferencia entre dos valores
consecutivos del rango.

Si el valor inicial es omitido, se supone que es 0. Si el incremento es
omitido, se supone que es 1.

\begin{table}
  \centering
  \begin{tabular}{ll}
    \toprule
      Rango                        & Valores que incluye \\
    \midrule
      \lstinline!range(9)!         & 0, 1, 2, 3, 4, 5, 6, 7, 8 \\
      \lstinline!range(3, 13)!     & 3, 4, 5, 6, 7, 8, 9, 10, 11, 12 \\
      \lstinline!range(3, 13, 2)!  & 3, 5, 7, 9, 11 \\
      \lstinline!range(11, 4)!     & Ningún valor \\
      \lstinline!range(11, 4, -1)! & 11, 10, 9, 8, 7, 6, 5 \\
    \bottomrule
  \end{tabular}
  \caption{Ejemplos de rangos.}
  \label{tbl:ejemplos-range}
\end{table}

Con los ejemplos de la tabla~\ref{tbl:ejemplos-range} quedará más claro.

Usando un incremento negativo, es posible hacer ciclos que van hacia
atrás:

\begin{lstlisting}
for i in range(10, 0, -1):
    print i
print 'Feliz anno nuevo!'
\end{lstlisting}

En general, el ciclo \lstinline!for! con rango se usa cuando el número
de iteraciones es conocido antes de entrar al ciclo.

\section{Salir de un ciclo}

Además de las condiciones de término propias de los ciclos
\lstinline!while! y \lstinline!for!, siempre es posible salir de un
ciclo en medio de una iteración usando la sentencia \lstinline!break!:
\begin{center}
  \begin{tikzpicture}[node distance=12ex, auto]
    \node (start) { };
    \node [decision, below of=start] (cond) { condición };
    \node [stmt, right of=cond, node distance=8em] (stmt1) { sentencia };
    \node [decision, below of=stmt1] (cond-break) { ¿salir? };
    \node [stmt, below of=cond-break] (stmt2) { sentencia };
    \node [below of=cond, node distance=30ex] (end) { };

    \path [conn] (start) -- (cond);
    \path [conn] (cond) -- (stmt1);
    \path [conn] (stmt1) -- (cond-break);
    \path [conn] (cond-break) --
                 node[very near start] { no }
                 (stmt2);
    \path [conn] (cond) -- (end);
    \path [conn] (stmt2) --
                 +(6em, 0) |-
                 ($ (start)!.5!(cond.north) $);
    \path [conn] (cond-break) --
                 node[very near start] { sí }
                 +(-8em, 0);
\end{tikzpicture}

\end{center}

Por ejemplo, en el programa para determinar si un número es primo o no,
la búsqueda de divisores puede ser terminada prematuramente apenas se
encuentra el primero de ellos:

\begin{lstlisting}
es_primo = True
for d in range(2, n):
    if n % d == 0:
        es_primo = False
        break
\end{lstlisting}

Es lógico es que el \lstinline!break! aparezca siempre dentro de un \lstinline!if!,
pues de otro modo el ciclo terminaría siempre en la primera iteración.

\section{Saltar a la siguiente iteración}

La sentencia \textbf{continue} se usa para saltar a la iteración
siguiente sin llegar al final de la que está en curso.

\begin{center}
  \documentclass{minimal}
\usepackage[pdftex,active,tightpage]{preview}
\usepackage[utf8]{inputenc}
\usepackage[spanish]{babel}
\usepackage{mathpazo}
\usepackage{tikz}
\usetikzlibrary{calc,shapes,arrows}
\PreviewEnvironment{tikzpicture}

\begin{document}
\tikzstyle{decision} = [
  diamond,
  very thick,
  draw=red!50!black!50,
  %fill=red!20, 
  aspect=2,
  %text badly centered,
  top color=white,
  bottom color=red!50!black!20,
]
\tikzstyle{stmt} = [
  rectangle,
  very thick,
  draw=blue!50!black!50,
  %fill=blue!20, 
  text centered,
  minimum height=5ex,
  minimum width=5em,
  top color=white,
  bottom color=blue!50!black!20,
]
\tikzstyle{node} = [
  circle,
  very thick,
  draw=orange!50!black!50,
  fill=orange!20,
  minimum size=6ex,
]
\tikzstyle{io} = [
  very thick,
  draw=green!50!black!50,
  trapezium,
  trapezium left angle=80,
  trapezium right angle=-80,
  %fill=green!20!black!10,
  %rounded corners,
  %minimum height=5ex,
  top color=white,
  bottom color=green!50!black!20,
  text centered,
  minimum height=5ex,
  minimum width=5em,
]
\tikzstyle{conn} = [very thick, draw=black!50, -latex']


\begin{tikzpicture}[node distance=12ex, auto]
    \node (start) { };
    \node [decision, below of=start] (cond) { condición };
    \node [stmt, right of=cond, node distance=8em] (stmt1) { sentencia };
    \node [decision, below of=stmt1] (cond-cont) { ¿continuar? };
    \node [stmt, below of=cond-cont] (stmt2) { sentencia };
    \node [below of=cond, node distance=30ex] (end) { };

    \path [conn] (start) -- (cond);
    \path [conn] (cond) -- (stmt1);
    \path [conn] (stmt1) -- (cond-cont);
    \path [conn] (cond-cont) --
                 node[very near start] { no }
                 (stmt2);
    \path [conn] (cond) -- (end);
    \path [conn] (stmt2) --
                 +(6em, 0) |-
                 ($ (start)!.5!(cond.north) $);
    \path [conn] (cond-cont) --
                 node[very near start] { sí }
                 +(6em, 0);
\end{tikzpicture}
\end{document}

\end{center}

Por ejemplo, el siguiente programa muestra el seno, el coseno y la
tangente de los números del 1 al 30, pero omitiendo los que terminan en
7:

\begin{lstlisting}
from math import sin, cos, tan
for i in range(1, 31):
    if i % 10 == 7:
        continue
    print i, sin(i), cos(i), tan(i)
\end{lstlisting}

