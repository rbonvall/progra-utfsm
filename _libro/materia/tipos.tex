\chapter{Tipos de datos}

Un \textbf{tipo de datos} es la propiedad de un valor que determina su
dominio (qué valores puede tomar), qué operaciones se le pueden aplicar
y cómo es representado internamente por el computador.

Todos los valores que aparecen en un programa tienen un tipo.

A continuación revisaremos los tipos de datos elementales de Python.
Además de éstos, existen muchos otros, y más adelante aprenderemos a
crear nuestros propios tipos de datos.

\section{Números enteros}

El tipo \textbf{int} (del inglés \emph{integer}, que significa «entero»)
permite representar números enteros.

Los valores que puede tomar un \lstinline!int! son todos los números
enteros: \ldots{} \lstinline!-3!, \lstinline!-2!, \lstinline!-1!,
\lstinline!0!, \lstinline!1!, \lstinline!2!, \lstinline!3!, \ldots{}

Los números enteros literales se escriben con un signo opcional seguido
por una secuencia de dígitos:

\begin{lstlisting}[language=py]
1570
+4591
-12
\end{lstlisting}

\section{Números reales}

El tipo \textbf{float} permite representar números reales.

El nombre \lstinline!float! viene del término
\href{http://es.wikipedia.org/wiki/Punto\_flotante}{\emph{punto flotante}}, que
es la manera en que el computador representa internamente los números
reales.

Hay que tener cuidado, porque los números reales no se pueden
representar de manera exacta en un computador. Por ejemplo, el número
decimal 0.7 es representado internamente por el computador mediante la
aproximación 0.69999999999999996. Todas las operaciones entre valores
\lstinline!float! son aproximaciones. Esto puede conducir a resultados
algo sorpresivos:

\begin{lstlisting}[language=py]
>>> 1/7 + 1/7 + 1/7 + 1/7 + 1/7 + 1/7 + 1/7
0.9999999999999998
\end{lstlisting}

Los números reales literales se escriben separando la parte entera de la
decimal con un punto. Las partes entera y decimal pueden ser omitida si
alguna de ellas es cero:

\begin{lstlisting}[language=py]
>>> 881.9843000
881.9843
>>> -3.14159
-3.14159
>>> 1024.
1024.0
>>> .22
0.22
\end{lstlisting}

Otra representación es la notación científica, en la que se escribe un
factor y una potencia de diez separados por una letra \lstinline!e!. Por
ejemplo:

\begin{lstlisting}[language=py]
>>> -2.45E4
-24500.0
>>> 7e-2
0.07
>>> 6.02e23
6.02e+23
>>> 9.1094E-31
9.1094e-31
\end{lstlisting}

Los dos últimos valores del ejemplo son, respectivamente,
$6.02\times 10^{23}$
(la \href{http://es.wikipedia.org/wiki/Constante\_de\_Avogadro}{constante de Avogadro}) y
$9.1094\times 10^{-31}$
(la \href{http://es.wikipedia.org/wiki/Electr\%C3\%B3n\#Propiedades}{masa del electrón} en kilogramos).

\section{Números complejos}

El tipo \textbf{complex} permite representar números complejos.
Los números complejos tienen una parte real y una imaginaria. La parte
imaginaria es denotada agregando una \lstinline!j! pegada inmediatamente
después de su valor:

\begin{lstlisting}
3 + 9j
-1.4 + 2.7j
\end{lstlisting}

\section{Valores lógicos}

Los valores lógicos \lstinline!True! y \lstinline!False! (verdadero y
falso) son de tipo \textbf{bool}, que representa valores lógicos.

El nombre \lstinline!bool! viene del matemático
\href{http://es.wikipedia.org/wiki/George\_Boole}{George Boole}, quien
creó un sistema algebraico para la lógica binaria. Por lo mismo, a
\lstinline!True! y \lstinline!False! también se les llama
\textbf{valores booleanos}. El nombre no es muy intuitivo, pero es el
que se usa en informática, así que hay que conocerlo.

\section{Texto}

A los valores que representan texto se les llama \textbf{strings}, y
tienen el tipo \textbf{str}.
Los strings literales pueden ser representados con texto entre comillas
simples o comillas dobles:

\begin{lstlisting}[language=py]
"ejemplo 1"
'ejemplo 2'
\end{lstlisting}

La ventaja de tener dos clases de comillas es que se puede usar una de
ellas cuando la otra aparece como parte del texto:

\begin{lstlisting}[language=py]
"Let's go!"
'Ella dijo "hola"'
\end{lstlisting}

Es importante entender que los strings no son lo mismo que los valores
que en él pueden estar representados:

\begin{lstlisting}[language=py]
>>> 5 == '5'
False
>>> True == 'True'
False
\end{lstlisting}

Los strings que difieren en mayúsculas y minúsculas
o en espacios en blanco
también son distintos:

\begin{lstlisting}[language=py]
>>> 'mesa' == 'Mesa'
False
>>> ' mesa' == 'mesa '
False
\end{lstlisting}

\section{Nulo}

Existe un valor llamado \textbf{None} (en inglés, «ninguno») que es
utilizado para representar casos en que ningún valor es válido, o para
indicar que una variable todavía no tiene un valor que tenga sentido.
El valor \lstinline!None! tiene su propio tipo, llamado
\lstinline!NoneType!, que es diferente al de todos los demás valores.
